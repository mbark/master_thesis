\documentclass[a4paper,11pt]{kth-mag}
\usepackage[T1]{fontenc}
\usepackage{textcomp}
\usepackage{lmodern}
\usepackage[utf8]{inputenc}
\usepackage[swedish,english]{babel}
\usepackage{csquotes}
\usepackage{listings}
\usepackage{enumitem}
\usepackage{courier}
\usepackage{epigraph}
\usepackage{modifications}
\usepackage{algorithm}
\usepackage{pifont}
\usepackage{numberedblock}
\usepackage{forest}
\usepackage{subcaption}
\usepackage{pgfplots}
\usepackage{tabu}
\usepackage{fancyvrb}
\usepackage[section]{minted}
\usepackage{standalone}
\usepackage[backend=biber]{biblatex}
\usepackage{hyperref}

\addbibresource{references.bib}

\title{Evaluating the effect of cardinality estimates on two state-of-the-art
  query optimizer's selection of access method}

\subtitle{Studying how MariaDB's and PostgreSQL's respective query optimizers
  select access method in correlation to the sample size used when estimating cardinality.}
\foreigntitle{En utvärdering av kardinalitetsuppskattningens påverkan på två
  state-of-the-art query optimizers val av metod för att hämta data}
\author{Martin Barksten}
\date{January 2016}
\blurb{Master's Thesis at NADA\\Supervisor: John Folkesson\\Examiner: Patric Jensfelt}
\trita{TRITA xxx yyyy-nn}

\begin{document}
\lstset{basicstyle=\ttfamily,breaklines=true}
\lstset{frame=lines}
\emergencystretch=1em
\pgfplotsset{width=10cm, compat=1.9}
\tabulinesep=1.2mm

\makeatletter
\g@addto@macro\@floatboxreset\centering
\makeatother

\newcommand{\json}[5]{
  \inputminted[breaklines, breakanywhere, fontsize=\footnotesize]{json}{#1}
  \captionof{listing}{The output when testing #2 with query #3, a statistics
    target of $#4$ and #5 repetitions.}
}
\newcommand{\clj}[1]{\mintinline{clojure}{#1}}
\newcommand{\sql}[1]{\mintinline{sql}{#1}}
\newenvironment{indexgraph}{
  \begin{tikzpicture}[scale=0.8]
    \begin{axis}[
      ybar,
      legend style={at={(0.5,-0.15)},
        anchor=north,legend columns=-1},
      symbolic x coords={ct,t,mt,mm,book,cmt,cmm,est,resamb},
      nodes near coords,
      ylabel={\#access methods},
      ytick=\empty,
      xtick=data]
} {
  \legend{MariaDB, PostgreSQL}
\end{axis}
\end{tikzpicture}
}

\newenvironment{indexplot}{
\begin{tikzpicture}[scale=0.8]
  \begin{axis}[
    ylabel={\#relations with varying access methods},
    xlabel={target statistic},
    legend style={at={(0.5,-0.2)},
      anchor=north,legend columns=-1},
    xtick={0,1,2},
    ytick={0, 1,2,3,4,5},
    xticklabels={$1$, $d$, $2d$},
    domain=0:2]
}{
  \legend{PostgreSQL, MariaDB}
\end{axis}
\end{tikzpicture}
}

\frontmatter
\pagestyle{empty}
\removepagenumbers{}
\maketitle
\selectlanguage{english}
\begin{abstract}
    This master thesis concern relational databases and their query optimizer's
sensitivity to cardinality estimates and the effect the quality of the estimate
has on the number of different access methods used for the same relation. Two
databases are evaluated --- PostgreSQL and MariaDB --- on a real-world dataset
to provide realistic results. The evaluation was done via a tool implemented in
Clojure and tests were conducted on a query and subsets of it with varying
sample sizes used when estimating cardinality.

The results indicate that MariaDB's query optimizer is less sensitive to
cardinality estimates and for all tests select the same access methods,
regardless of the quality of the cardinality estimate. This stands in contrast
to PostgreSQL's query optimizer which will vary between using an index or doing
a full table scan depending on the estimated cardinality. Finally, it is also
found that the predicate value used in the query affects the access method used.
Both PostgreSQL and MariaDB are found sensitive to this property, with MariaDB
having the largest number of different access methods used depending on
predicate value.

\end{abstract}
\clearpage
\begin{foreignabstract}{swedish}
    \input{Content/0_2_foreign_abstract}
\end{foreignabstract}
\clearpage
\tableofcontents*
\clearpage
\listoffigures
\mainmatter{}
\pagestyle{newchap}

\chapter*{Glossary}\label{chap:glossary}
    The terminology of databases often varies across literature and database
vendors, this section therefore defines the terms used in this thesis. When
it is the case that something commonly goes by other names as well, an attempt
has been made to include these names as well.

\subsection*{Database}
No distinction is made between the database and the Database Management Systems
(DBMS) in this thesis and the terms are used interchangeably.

\subsection*{Data}
The \textit{data} in a database is the values stored in the rows and columns of
the database. This does not include additional information stored in the
database such as indexes.

\subsection*{Dataset}
A \textit{dataset} is all information stored in the database, including both
data and additional information such as indexes, primary and foreign keys etc.

\subsection*{Query optimizer}
The terms query optimizer and optimizer are used interchangeably throughout the
thesis. If an optimizer of some other kind is used this will be made explicit.

\subsection*{Host variable}
A \textit{host variable} is a variable declared in the program in which the SQL
statement is embedded~\cite[p. 151]{chamberlin_1998_complete_acgtdud}. Host
variables are identified by the fact that they begin with a colon. An example of
a host variable is \sql{:HEIGHT} in Figure~\ref{fig:sql:hostvar}.

\begin{figure}[ht]
  \begin{minted}[breaklines]{sql}
    SELECT  NAME
    FROM    PERSONS
    WHERE   HEIGHT = :HEIGHT
  \end{minted}
  \caption[A query with a host variable]{A simple query using a host
    variable.}\label{fig:sql:hostvar}
\end{figure}

\subsection*{Predicate}
In order to get the data you need, you must be able to specify what conditions
the data should fulfill to be relevant, this is done by specifying predicates.
To illustrate, consider Figure~\ref{fig:sql:predicate} (example taken
from~\cite{lahdenmaki_2005_relational_rdidatodossea}).

\begin{figure}[ht]
  \begin{minted}[breaklines]{sql}
    WHERE   SEX = 'M'
    AND
    (WEIGHT = 90
    OR
    HEIGHT > 190)
  \end{minted}
  \caption[The \sql{WHERE} clause of a query containing three predicates and two
  compound predicates]{The \sql{WHERE} clause an SQL query containing three
    predicates and two compound predicates.}\label{fig:sql:predicate}
\end{figure}

The \sql{WHERE} clause in Figure~\ref{fig:sql:predicate} contains three
predicates:
\begin{enumerate}
\item \sql{SEX = 'M'}
\item \sql{WEIGHT = 90}
\item \sql{HEIGHT > 190}
\end{enumerate}

A \textit{compound predicate} is two or more predicates that are tied together
in the form of \sql{AND}, \sql{OR} or other similar operators. The \sql{WHERE}
clause in Figure~\ref{fig:sql:predicate} can be considered to have two different
compound predicates:
\begin{enumerate}
\item \sql{WEIGHT = 90 OR HEIGHT > 190}
\item \sql{SEX = 'M' AND (WEIGHT = 90 OR HEIGHT > 190)}
\end{enumerate}

\subsection*{Index slice}
The term \textit{index slice} comes
from~\cite{lahdenmaki_2005_relational_rdidatodossea} and is defined as the
number of index rows that need to be read for a predicate; the thinner the slice
the less amount of index rows that need to be read, and consequently the number
of reads to the table.

The thickness of the index slice will depend on the number of \textit{matching
  columns} --- the number of columns that exist both in the \sql{WHERE}
clause and the index. To illustrate why, consider the query in
Figure~\ref{fig:sql:indexslice}.

\begin{figure}[ht]
  \begin{minted}[breaklines]{sql}
    WHERE   WEIGHT = 90
    AND
    HEIGHT > 190
  \end{minted}
  \caption[The \sql{WHERE} clause of a query with two potential matching
  columns]{The \sql{WHERE} clause of a query with two potential matching
    columns.}\label{fig:sql:indexslice}
\end{figure}

If there exists an index on only \sql{HEIGHT}, no values for \sql{WEIGHT} can be
discarded in the index slice. If an index is added for \sql{WEIGHT}, the
thickness of the index slice will decrease as only values fulfilling both the
\sql{HEIGHT} and \sql{WEIGHT} requirements remain.

\subsection*{Indexable predicate}
A \textit{indexable predicate} is a predicate that can be evaluated when the
index is accessed (allowing a matching index scan)~\cite{2014_summary_sopp,
  2013_ibm_ikcianp}. Revisiting the example from earlier, both of the predicates
in Figure~\ref{fig:sql:indexslice} are examples of indexable predicates.

\subsection*{Matching predicate}
A \textit{matching predicate} is an indexable predicate with the corresponding
necessary indexes~\cite{2013_ibm_ikcianp}. In Figure~\ref{fig:sql:indexslice}
both predicates are indexable and would be matching if there exists an index for
\sql{WEIGHT} and \sql{HEIGHT} respectively.

\subsection*{Non-indexable predicate}
A \textit{difficult predicate} (also sometimes called \textit{nonsearch
  arguments}, \textit{index suppression}, \textit{difficult predicate}) is the
opposite of an indexable predicate, and can as a consequence not define the
index slice~\cite{lahdenmaki_2005_relational_rdidatodossea}. What predicates are
non-indexable varies from database to database, but a typical example of one can
be seen in Figure~\ref{fig:sql:nonindexable}.

\begin{figure}[ht]
  \begin{minted}[breaklines]{sql}
    COL1 NOT BETWEEN :hv1 AND :hv2
  \end{minted}
  \caption[An example of a non-indexable predicate]{A example of a commonly used
    non-indexable predicate.}\label{fig:sql:nonindexable}
\end{figure}

\subsection*{Boolean term predicate}
A \textit{boolean term predicate} (BT predicate) is one that can reject a row
because it does not match the
predicate~\cite{lahdenmaki_2005_relational_rdidatodossea}. Conversely a non-BT
predicate is a predicate that cannot reject a row. Non-BT predicates are
typically the result of using \sql{OR}. To illustrate when a predicate is BT
respectively non-BT consider, assume there is an index \texttt{(A, B)} on
\sql{TABLE} and consider Figure~\ref{fig:sql:btpredicate} and
Figure~\ref{fig:sql:nonbtpredicate}.

\begin{figure}[ht]
  \begin{minted}[breaklines]{sql}
    SELECT  A, B
    FROM    ATABLE
    WHERE   A > :A
    AND
    B > :B
  \end{minted}
  \caption[A query containing a BT predicate]{A query with a BT
    predicate.}\label{fig:sql:btpredicate}
\end{figure}

\begin{figure}[ht]
  \begin{minted}[breaklines]{sql}
    SELECT  A, B
    FROM    ATABLE
    WHERE   A > :A
    OR
    B > :B
  \end{minted}
  \caption[A query containing no BT predicates]{An query with no BT
    predicates.}\label{fig:sql:nonbtpredicate}
\end{figure}

For the query in Figure~\ref{fig:sql:btpredicate} if the first predicate \sql{A
  > :A} evaluates to false for a row the row can be rejected instantly, making
it a BT predicate. For the query in Figure~\ref{fig:sql:nonbtpredicate} on the
other hand it might be the case that \sql{B > :B} evaluates to true even if
\sql{A > :A} does not, making both predicates non-BT predicates.

\subsection*{Index screening}
A column may be in both the \sql{WHERE} clause and the index, yet be unable to
participate in defining the index slice due to other
reasons~\cite{lahdenmaki_2005_relational_rdidatodossea}. Even if this is the
case the column may still be able to reduce the amount of reads to the table
anyway. A column fulfilling these criteria is a \textit{screening column} as the
presence of it in the index allows not reading from the table. The process of
determining which columns might fulfill this is called \textit{index screening}.

\subsection*{Cardinality}
The \textit{cardinality} is the number of distinct values for a column, or
combination of columns~\cite{lahdenmaki_2005_relational_rdidatodossea}. The
cardinality of the data is usually used when the query optimizer estimates the
cost of different access paths.

\subsection*{Statistics target}
The \textit{statistics target} is used to refer to the amount of statistics
stored when analyzing a table, and thus by extension the sample size used when
estimating the cardinality of the relations. What the statistics target is
varies between database implementations.

\subsection*{Filter factor}
The \textit{filter factor} specifies what proportion of the rows that    satisfy
the conditions in a predicate~\cite{lahdenmaki_2005_relational_rdidatodossea}.
The filter factor can be seen as the selectivity of a predicate and the lower it
is, the more the number of rows that are filtered out by a predicate. For a
predicate such as \sql{HEIGHT = :HEIGHT} there are three ways to talk about
filter factor:
\begin{itemize}
\item The \textit{value specific filter factor} is the filter factor for one
  specific value of \sql{:HEIGHT};
\item The \textit{average filter factor} is the average value for all value
  specific filter factors;
\item And the \textit{worst-case filter factor} is the highest possible filter
  factor for a given value of \sql{:HEIGHT}
\end{itemize}

\subsection*{Access path}
The query optimizers output is an \textit{access path}, which is an abstract
representation of the path to access the data.

\subsection*{Query plan}
The \textit{query plan} will be used to refer to the concrete representation
given by an database to describe the underlying access path.

\subsection*{Execution plan}
The \textit{execution plan} corresponds to an access path but describes how to
physically access the data.

\subsection*{Relation}
The words \textit{relation} and \textit{table} are synonymous and used
interchangeably throughout the thesis.
\chapter{Introduction}\label{chap:introduction}
    \epigraph{The person who gave us this book told us that the book describes a secret technology called a database.\\
We hear that the database is a system that allows everyone to share, manage, and use data.}{\textit{The King of Kod, from \cite[p. 6]{takahashi_2009_manga_tmgtd}}}

If you want to save data, you need a database. And almost every computer program need to save some form of data, consequently requiring them to use a database. The trend is also going towards us generating more and more data, putting higher strain on the databases and requiring better performance. To improve and develop databases is therefore a topic of much relevance in today's society.

One key component of databases is the query optimizer, the part of the database that analyses the users query and finds the optimal path to access it. Or rather, theoretically it finds the optimal path. Work has been done improving query optimizers since the early '70s \cite{chaudhuri_1998_overview_aooqoirs}, yet optimizers often select a bad access path, causing slow queries \cite{leis_2015_how_hgaqor}.

Guy Lohman identifies the cardinality estimate of the data to be main cause for bad plans:

\textit{``The root of all evil, the Achilles Heel of query optimization, is the estimation of the size of intermediate results, known as cardinalities''}

The estimates can often be wrong by several orders of magnitude \cite{lohman_query_iqoap}. These incorrect estimates then propagate through the query and grow at an exponential rate \cite{ioannidis_1991_propagation_otpoeitsojr}, making the query optimizer base its decisions on completely false grounds.

The topic of improving the estimations has seen some study, yet the evaluation of new methods is often done on data that is easy to estimate, being uniformly distributed. It is only recently that a study has been done to analyze the performance of the optimizer end-to-end on complex real-world data \cite{leis_2015_how_hgaqor}. The study found the optimizer to perform unnaturally well on the typically tested uniform data.

This thesis will therefore aim to provide further insight into performance of query optimizers by studying a previously unstudied metric and analysing the performance of state-of-the-art optimizers. The performance evaluation will be conducted on both the ``easy'' uniform data and complex real-world data.

\section{Problem statement}
In this thesis two open-source state-of-the-art databases are evaluated: MariaDB \cite{mariadb_m} and PostgreSQL \cite{postgresql_ptwmaosd}. One real-world dataset containing a large amount of data and with a complex schema and setup will be used in the evaluation. The database will be analyzed to measure the performance of the query optimizer in order to answer the question:

\textit{How much effect does the cost estimation have on the query optimizers selection of indexes during the join enumeration?}

The metric studied to evaluate this will be the size of the ambiguous indexes for a query. This metric gives a good indication to the effect varying cost estimations has on the plan selected by the query optimizer. For a more in-depth description of how the metric is measured, see \ref{sec:choiceofmethod}.

The exact setup and metrics of the dataset is described in Section \ref{sec:benchmark}. A motivation to why the databases are used is described in Section \ref{sec:choiceofdatabases}.

\section{Purpose} \label{sec:purpose}
Query optimizers make bad cardinality estimates, and as a consequence bad cost estimations of access paths \cite{leis_2015_how_hgaqor} – but how much does this affect the actual index selection process? Even though the errors may be large, they may not be large enough to actually influence errors in the index selection.

There are three steps to the optimization – search space expansion, cost estimation and join enumeration (more about those in Section \ref{sec:queryopt}) – this thesis will focus on measuring what effect bad performance in the first two steps has on the third and final one. The study will be done by identifying ambiguous indexes to see both how common they are, and of which size the set of indexes is. These two metrics will give a good indication to what effect the bad performance has on the final step of optimization.

Studying this is of relevance for the following reasons:
\begin{enumerate}
    \item\label{item:purpose:tool} The tool developed for evaluation can be used in the future to measure the performance of query optimizers;
    \item\label{item:purpose:steps} The evaluation will provide insight into what steps in the optimization process produce bad access paths;
    \item\label{item:purpose:data} The performance of query optimizers has not seen much study using actual real-world data;
    \item\label{item:purpose:performance} The actual performance of the databases right now will be evident;
    \item\label{item:purpose:compare} Since both databases compared are open-source, one performing better may guide development for the other.
\end{enumerate}

The primary interest for academia is probably reasons \ref{item:purpose:tool}, \ref{item:purpose:steps} and \ref{item:purpose:data} whereas database vendors might be more interested in \ref{item:purpose:performance} and \ref{item:purpose:compare}. This thesis will thus provide a foundation for further improving query optimizers and that is of relevance for everyone who uses a database.

\section{Outline}
Below is a brief outline of the chapters in the report and what can be expected to be found in each:

\begin{itemize}
    \item The \nameref{chap:introduction} chapter gives an introduction to the subject of query optimizer, the problem statement discussed, the purpose of the thesis and why it is novel and relevant.
    \item The \nameref{chap:relatedwork} chapter contains an overview of what previous and relevant work has been done in the area of improving and evaluating the performance of query optimizers.
    \item The \nameref{chap:theory} chapter gives a background and the information necessary to understand the thesis. The chapter begins with a background on how a modern query optimizer works. It then continues with a description of the individual characteristics of the databases analysed: MariaDB and PostgreSQL. It also defines some important terms used throughout the thesis.
    \item The \nameref{chap:method} chapter describes the performance tests that were run and how they were implemented. It also describes more in-detail the data used for the databases, the database configurations used and the environment the tests were run in.
    \item The \nameref{chap:results} chapter displays the results from the performance test in the form of graphs. It also gives some brief commentary on them.
    \item The \nameref{chap:discussion} chapter discusses the performance of the query optimizers and the consequences of it, as well as the validity of the results. It also answers the problem statement and provides suggestions for future research.
\end{itemize}
\chapter{Related work}\label{chap:relatedwork}
    Improving the query optimizer is a topic naturally tied to that of evaluating
the query optimizer's performance. In spite of this, the optimizer's performance
has not seen much study, while improving them on the other hand has. This
section will start with a section about some of the more recent evaluations that
have been done and then continue with a section about some improvements done to
the query optimizer relating cardinality estimation. A final section describes
some implementations to improve the performance of relational operators and make
them more resistant to bad plans.

\section{Evaluating the query optimizer's performance}
In~\cite{leis_2015_how_hgaqor} Leis et.\ al.\ perform what they claim is:

\textit{``the first end-to-end study of the join ordering problem using a
  real-world data set and realistic queries''}.

In the study they create the a database setup based on the Internet Movie
Database (IMDb), create a set of realistic queries for it and call it the Join
Order Benchmark (JOB). Using this benchmark they then measure how PostgreSQL,
HyPer and three unnamed commercial databases perform in terms of cardinality
estimates, cost modelling and general performance. They also compare the results
to TPC-H, the database setup previously mostly used for evaluation and show that
the PostgreSQL optimizer performs unrealistically well for TPC-H because of the
uniform data distribution. The results of their study show that relational
databases produce large estimation errors and that primarily the cardinality
estimate is to blame.

Another article that has evaluated optimizers
is~\cite{wu_2013_predicting_pqetaocmru} where Wu  et\ al.\ analyzed if the
optimizer's cost model can be used to estimate the actual run-time of the query.
They find that the optimizer consistently makes bad cost estimates, but show
that for analyzing actual run-time a more costly and precise analysis can be
conducted on the selected access path.

Evaluating a query optimizer's cardinality estimation is often done through a
comparison with the actual cardinality. Calculating the exact cardinality can
however be very costly for complex queries and datasets. A more efficient method
that can find the exact cardinalities by studying a subset of all expressions is
presented in~\cite{chaudhuri_2009_exact_ecqofot}.

One novel way of studying and analyzing the plans chosen by the query optimizer
is a tool called Picasso, which allows query plans to be visualized as
two-dimensional diagrams~\cite{haritsa_2010_picasso_tpdqov}. The tool provides a
visualisation of the performance across the entire query plan space, thus
providing another way of analysing queries or query optimizers.

An example of a visualisation done with Picasso can be seen in
Figure~\ref{fig:picasso}. The colored regions represent a specific execution
plan, the X and Y-axis represent the selectivity for the attributes
\sql{SUPPLIER.S_ACCTBAL} and \sql{LINEITEM.L_EXTENDEDPRICE} respectively. The
percentages in the legend correspond to the area covered by each plan.

\begin{figure}[ht]
  \includegraphics[scale=0.6]{Images/Picasso.png}
  \caption[A query plan visualisation done by Picasso]{A query plan
    visualisation done by Picasso, image taken
    from~\cite{haritsa_2010_picasso_tpdqov}.}\label{fig:picasso}
\end{figure}

Finally, on a more practical note Lahdenmäki  et\ al.\ describe of how to
identify queries where the selected access path is bad, and how to solve the
problem~\cite{lahdenmaki_2005_relational_rdidatodossea}. The book gives a
thorough introduction to many important aspects of the database and the chapter
``Optimizers Are Not Perfect'' focuses on incorrect cardinality estimates and
other common query optimizer errors.

\section{Bad statistics and cardinality estimates}
In~\cite{ioannidis_1991_propagation_otpoeitsojr} Ionnidis  et\ al.\ develop a
framework to study how cardinality estimate errors propagate in queries. Their
results indicate that the error increases exponentially with the number joins.

There are two different methods of improving how optimizers handle cardinality
estimation:
\begin{itemize}
\item Reducing the effect of incorrect estimations by making plans more
  \textit{robust}, which means they perform better over large regions of the
  search space;
\item Or by improving the estimations.
\end{itemize}

\subsection{Improving robustness}
Harish et\ al.\ present a way to make plans more robust by allowing the
optimizer to select the most robust plan that is not ``too slow'' compared to
the calculated optimal path.~\cite{harish_2008_identifying_irptpdr}. They
develop an external tool for this purpose and find that the tool indeed does
improve performance by reducing the effect of selectivity errors.

A similar study is done by Abhirama et\ al.\ but they implement the selection
directly in the PostgreSQL query
optimizer~\cite{abhirama_2010_stability_otsopcatcops}. Their results agree with
those found by Harish  et\ al.\ in that the performance is improved. The results
they present show that robut plans often reduced the adverse effects of
selectivity errors by more than two-thirds, while only providing a minor overhead in
terms of time and memory.

\subsection{Improving cardinality estimates}\label{sec:improving_cardinality_estimates}
The most studied problem of cardinality estimate is that of finding the right
balance between calculation time, memory overhead and correctness. One common
method used in current state-of-the-art databases is histograms that assume
attributes are independent of each other, an assumption that often is not
correct~\cite{ioannidis_2003_history_thoha}. Recent studies have been done to
find alternatives method that do not assume independence.

Tzoumas et\ al.\ present one method that instead of the usual one-dimensional
statistical summary, saves it as a small, two-dimensional
distribution~\cite{tzoumas_2011_lightweight_lgmfsewia}. Their results show an
small overhead, and an order of magnitude better selectivity estimates.

In~\cite{yu_2013_cs2_candsfqea} Yu et\ al.\ develop a  method called
\textit{Correlated Sampling} that does not sample randomly, but rather save
correlated sample tuples that retain join relationships. They further develop an
estimator, called reverse estimator, that use correlated sample tuples for query
estimation. Their results indicate that the estimator is fast to construct and
provides better estimations than existing methods.

In~\cite{vengerov_2015_join_jsestfc} Vengerov et\ al.\ once again study
Correlated Sampling, but improve on it by allowing it to only make a single pass
over the data. They compare the algorithm to two other sampling approaches
(independent Bernoulli Sampling and End-Biased Sampling, which is described
in~\cite{estan_2006_end_esfjce}) and find Correlated Sampling to give the best
estimates in a large range of situations.

\section{Improving operators}
In~\cite{muller_2015_cache_cahis} Müller  et\ al.\ study the two implementations
of relational operators: hashing and sorting (more about these in
Section~\ref{sec:opimpl}). Their study find that the two paradigms are in terms
of cache efficiency actually the same and from this observation develop a
relational aggregation algorithm that outperform state-of-the-art methods by up
to 3.7 times.

A problem described in~\cite{leis_2015_how_hgaqor} is that the optimizer tends
to pick nested-loop joins (more about these in Section~\ref{sec:nestedloopjoin})
even though they provide a high risk but only a very small payoff.
In~\cite{graefe_2011_generalized_agja} Goetz Graefe provides a generalized join
algorithm that can replace both merge joins and hash joins in databases, thus
avoiding the danger of mistaken join algorithm choices during query
optimization.

\chapter{Theory}\label{chap:theory}
    \epigraph{An SQL query walks into a bar and sees two tables.\\
He walks up to them and asks ``Can I join you?''}{\textit{– Source: Unknown, from~\cite{join_tjo}}}

In this chapter a background is given to relational databases with more focus on
the areas of interest for this thesis. The first section will give a high-level
introduction to relational databases and how they work in general. Following
this is a section with a more in-depth description of indexes. After this comes
the final section covering the query optimizer, detailing how it works, how it
can be monitored and its limitations.

\section{Relational databases}
A database is a computerized record-keeping system, a way to save computerized
data~\cite[p. 6]{date_2003_introduction_aitds}. The data stored in the database
can then be accessed and modified by the user. Accessing and modifying the data
is typically done through a layer of software called the database management
system, which provides a method for accessing and modifying the data.

A database stores data in the form of rows in different tables. These rows are
also sometimes referred to as tuples. In a relational database these tuples can
then have relations between each other in the form of for example ``Tuple A has
one or more of Tuple B''.

The following sections will describe some components of the database which are
of the most relevant for this thesis.  First comes a section describing the most
common method to access data in a relational database, SQL, after this is is a
section about how the query described by SQL is executed and finally a section
about one of the most fundamental operations in SQL – the join operation.

\subsection{SQL}
SQL, or Structured Query Language in full, is the computer language most
commonly used to query and modify the database, it is formally defined by
ISO/IEC 9075~\cite[p. 29]{garcia-molina_2002_database_dstcb}\cite{iso_i9itdlsp1f}. The language came
from research into manipulating databases in the early '70s and it is now one of
the most popular database languages in existence~\cite{sql_s|cl}.

\subsection{Query execution}
The execution of a query in the form of an SQL statement is split into four phases~\cite{selinger_1979_access_apsiardms}:
\begin{enumerate}
    \item \textit{Parsing}, in which the input text is transformed into query blocks;
    \item \textit{Optimization}, in which an optimized way to access the data is found, called an \textit{access path};
    \item \textit{Code generation}, in which the access path is transformed a way to execute it, an \textit{execution plan};
    \item And \textit{execution}, when the code is executed;
\end{enumerate}

The \textit{parsing}, \textit{code generation} and \textit{execution phases} are
all fairly trivial compared to the \textit{optimization}. The optimization
process is also the phase that has the potentially most effect on the execution
time for the query. The query optimization process is performed by the query
optimizer, which is described in more detail in Section~\ref{sec:queryopt}.

\subsection{The join operation}
One of the most fundamental operations in SQL is that of joining two tables. An
example of an inner join on the tables \sql{EMPLOYEES} and \sql{DEPARTMENTS} can
be found in Figure~\ref{fig:sql:joinop}.

\begin{figure}[ht]
\begin{minted}[breaklines]{sql}
SELECT *
FROM   EMPLOYEES, DEPARTMENTS
WHERE  EMPLOYEES.DEPARTMENT_ID = DEPARTMENTS.DEPARTMENT_ID
AND    EMPLOYEES.NAME = 'John';
\end{minted}
\caption[An example of an SQL query]{An SQL query that will find the all employees by the name of John and info about their department.}\label{fig:sql:joinop}
\end{figure}

There are four kinds of joins typically supported in databases. To illustrate
this assume we have the following \texttt{Join(A, B)}, where \texttt{A} and
\texttt{B} are tables and \texttt{Join} is one of the join operations.
\begin{itemize}
    \item An inner join will return all rows in common between \texttt{A} and \texttt{B};
    \item A left outer join will return all rows in \texttt{A} and all common rows in \texttt{B};
    \item A right outer join will return all rows in \texttt{B} and all common rows in \texttt{A};
    \item And a full outer join will return all rows in \texttt{A} and all rows in \texttt{B}.
\end{itemize}
See Figure~\ref{fig:vennjoin} for a visualization of the joins using Venn diagrams.

\begin{figure}[ht]
\includegraphics[scale=0.4]{Images/SQL-Join-Venn-Diagrams.jpg}
\caption[Illustration of join operations using Venn diagrams]{The four join operations illustrated using Venn diagrams, image taken from~\cite{brian_2014_better_bqj}.}\label{fig:vennjoin}
\end{figure}

\subsection{Implementation of operators}\label{sec:opimpl}

Operators can be divided into three classes:
\begin{enumerate}
    \item\label{item:classop:sort} \textit{Sorting-based} methods;
    \item\label{item:classop:hash} \textit{Hash-based} methods;
    \item\label{item:classop:index} And \textit{index-based} methods.
\end{enumerate}
In general index-based methods are variations of class~\ref{item:classop:sort}
or~\ref{item:classop:hash} that utilize indexes to speed up parts of the
algorithm. Most notably when there exists an index, for example B-trees, that
allows the data to be accessed sorted joins can be done very efficiently.

Furthermore the algorithms for the operators can be divided by the number of passes the algorithm does:
\begin{enumerate}
    \item \textit{One-pass algorithms} read the data from disk only once;
    \item \textit{Two-pass algorithms} reads the data once, processes it and saves it in some way before doing another pass;
    \item And algorithms that do three or more passes, which are essentially generalizations of two-pass algorithms.
\end{enumerate}

There are several operators to implement in a database, but the most relevant
algorithms for this thesis are those implementing the join operator.
Implementation of the join operator is typically done with three fundamental
algorithms: nested loop join, merge join and hash join~\cite{postgresql_pd9p}.
Many databases support more join algorithms, but they are typically variations
of one of these three algorithms.

For the descriptions of the algorithms assume we have a query joining the tables
\texttt{A} and \texttt{B}, \texttt{Join(A, B)}. The first table of the join
operation, \texttt{A}, is referred to as the \textit{outer table} and the second
table, \texttt{B}, as the \textit{inner table}.

\subsubsection{Nested loop join}\label{sec:nestedloopjoin}
A nested loop join is essentially two nested loops, one over the outer table and
inside it one over the inner table~\cite[p. 718-722]{garcia-molina_2002_database_dstcb}. The nested loop join is a bit of a
special case as it is not necessarily of any of the classes. The number of
passes it does can also be considered to be ``one-and-a-half'' as the outer
table's tuples is read only once, while the inner table's tuples are read
repeatedly.

If an index exists for the inner table the nested loop join could be considered
to be of class~\ref{item:classop:index} and is then called a \textit{index
  nested loop join}.

\subsubsection{Merge join}
A merge join (sometimes also called \textit{sort-merge join}) is a variation of
a nested loop join that requires both tables to be sorted. The two tables can
then be scanned in parallel, allowing each row to only be scanned
once~\cite[p. 723-730]{garcia-molina_2002_database_dstcb}. The sorting of the
tables can be achieved through an explicit sort step or through an index.

A merge join can be considered to be a two-pass algorithm of class~\ref{item:classop:sort}.

\subsubsection{Hash join}
There are several types of hash joins but the general principle remains the
same: build a hash table for the outer table, then scan the inner table to find
rows that match the join condition~\cite[p. 732-738]{garcia-molina_2002_database_dstcb}.

Hash joins are two-pass algorithms of class~\ref{item:classop:hash}.

\section{Indexes}
This section will cover the basics behind indexes as described by Ramakrishnan
et.\ al.\ in~\cite[Ch. 8]{ramakrishnan_2003_database_dms}.

An index is a data structure that allows data stored in the database to be accessed quicker through some retrieval operations. The data stored in the index is called the \textit{data entry} and the value that is indexed – the value in the column – is called the \textit{search key}. There are three alternatives for what to store as the data in the data entry:

\begin{enumerate}
    \item\label{item:indexes:alt1} A data entry is a the actual data saved;
    \item\label{item:indexes:alt2} A data entry is a pair containing a search key and record id;
    \item\label{item:indexes:alt3} A data entry is a pair containing a search key and a list of record ids corresponding to the key.
\end{enumerate}

Which alternative is used depends on how the index is created and what kind of an index it is.

\subsection{Composite indexes}
Composite indexes are indexes containing more than one fields. A composite index can support a broader range of queries than a normal index and since they also contain more columns, they contain more information about the data saved.

\subsection{Clustered index}
A clustered index is an index on a column which is sorted in the same way as the
index; otherwise it is an unclustered index. An index using
Alternative~\ref{item:indexes:alt1} is sorted by definition, whereas
Alternative~\ref{item:indexes:alt2} and~\ref{item:indexes:alt3} require the data
stored to be sorted.

\subsection{Data structures}
The two most common indexes used are \textit{hash-based indexes} and \textit{tree-based indexes}. Below is a more detailed description of both.

\subsubsection{Hash-based indexes}
A hash-based index is implemented as a hash table, mapping the hashed value of a search key to a bucket containing one or more values. To search in the index the search key is hashed and the corresponding bucket is identified, all values in the bucket are then examined to identify the matching one.

\subsubsection{Tree-based indexes}
Tree-based indexes save the data as hierarchical sorted trees where the leaf nodes contain the values. To find a value the search starts at the root and all non-leaf nodes direct the search toward the correct leaf node. In practice the trees are often implemented as $B^{+}$-trees, which is a data structure that ensure that paths from the root to a leaf node are of the same length~\cite{comer_1979_ubiquitous_ub}. The efficiency of a B-tree index depends on the number of levels the B-tree has~\cite[p. 645]{garcia-molina_2002_database_dstcb}.

\section{The query optimizer}\label{sec:queryopt}
In order to access the data in an as efficient way as possible, the query is optimized by a built-in tool in the databases called the query optimizer. Below is a description of the fundamental operations performed by query optimizer, taken mostly from C., Surajits article on the topic~\cite{chaudhuri_1998_overview_aooqoirs}.

Query evaluation is handled by two components: the \textit{query optimizer} and the \textit{query execution engine}. The input to the query optimizer is a parsed representation of the SQL query and the output is an execution plan that the query execution engine then performs.

In order for the query optimizer to find an access path it must be able to:
\begin{enumerate}
    \item Expand the \textit{search space} to find all access paths that are valid transformations of the query;
    \item Perform a \textit{cost estimation} for each access path to calculate its cost;
    \item And finally \textit{enumerate} the access paths to find which is the best.
\end{enumerate}

A good query optimizer is one that does not cause too much overhead in the query execution in calculating the access path, while still finding a good access path. In order to do this each step must fulfill the criteria:
\begin{enumerate}
    \item The search space includes plans with a low cost;
    \item The cost estimation is accurate;
    \item And the enumeration algorithm is efficient.
\end{enumerate}

\subsection{Expanding the search space}
The first task of the query optimizer is that of taking the original search space containing just the original query, and expanding it through transformation rules. The expansion will thus generate a larger search space containing valid permutations of the join order. The output is a set of \textit{operator trees}, which are binary trees where nodes represent operations and leaves values.

There are multiple rules that can be applied, most of which are complex and work only under some specific conditions. The most relevant rules for index-selection are described below, for a more in-depth description see~\cite{chaudhuri_1998_overview_aooqoirs}.

\subsubsection{Join reordering}
One important rule is that both inner join and full join are:
\begin{itemize}
    \item commutative: \texttt{Join(R1, R2)} is equivalent to \texttt{Join(R2, R1)};
    \item Associative, \texttt{Join(R1, Join(R2, R3))} is equivalent to \texttt{Join(Join(R1, R2), R3)}.
\end{itemize}
This means that the joins can be grouped and reordered as the optimizer finds best. Another consequence of this is that the operators that can be seen as a single node with many children, see Figure~\ref{fig:groupop} for an example.

\begin{figure}[ht]
\begin{subfigure}[b]{0.5\linewidth}
\centering
\begin{forest}
[\texttt{JOIN}
    [\texttt{JOIN}
        [\texttt{JOIN}
            [\texttt{R}]
            [\texttt{JOIN}
                [\texttt{S}]
                [\texttt{T}]]]
        [\texttt{U}]]
    [\texttt{JOIN}
        [\texttt{V}]
        [\texttt{W}]]]
\end{forest}
\caption{\label{fig:groupop:a}}
\end{subfigure}
\begin{subfigure}[b]{0.5\linewidth}
\centering
\begin{forest}
[\texttt{JOIN}
    [\texttt{JOIN}
        [\texttt{R}]
        [\texttt{S}]
        [\texttt{T}]]
    [\texttt{U}]
    [\texttt{V}]
    [\texttt{W}]]
\end{forest}
\caption{\label{fig:groupop:b}}
\end{subfigure}
\caption[An example of how operators can be grouped into a single node]{The \texttt{JOIN} operators are assumed to be associative and commutative, allowing Figure~\ref{fig:groupop:a} to be transformed into Figure~\ref{fig:groupop:b}, example taken from~\cite[p. 791]{garcia-molina_2002_database_dstcb}.}\label{fig:groupop}
\end{figure}

\subsubsection{Pushing operations up and down the tree}
Another fundamental rule used by optimizers is that of pushing an operator down the expression tree in order to reduce the cost of performing it~\cite[p. 768-792]{garcia-molina_2002_database_dstcb}. For the example the selection operators tend to reduce the size of the relations, meaning pushing them down as far down the tree as possible is beneficial.

Another rule that can be applied is to pull an operator up the tree, in order to then be able to push it down again to reduce the size of more relations. See Figure~\ref{fig:pushop}, which illustrate how pulling a selection up the tree allows it to then be pushed down more branches.

The conditions for when these rules are applicable naturally varies a lot depending on the operator and it is beyond the scope of this thesis to list them all. See~\cite[p. 768-779]{garcia-molina_2002_database_dstcb} for an in-depth description of the rules and conditions.

\begin{figure}[ht]
\begin{subfigure}[b]{0.5\linewidth}
\centering
\begin{forest}
[\texttt{starName, studioName}
    [\texttt{JOIN}
        [\texttt{year = 1996}
            [\texttt{Movies}]]
        [\texttt{StarsIn}]]]
\end{forest}
\caption{\label{fig:pushop:a}}
\end{subfigure}
\begin{subfigure}[b]{0.5\linewidth}
\centering
\begin{forest}
[\texttt{starName, studioName}
    [\texttt{JOIN}
        [\texttt{year = 1996}
            [\texttt{Movies}]]
        [\texttt{year = 1996}
            [\texttt{StarsIn}]]]]
\end{forest}
\caption{\label{fig:pushop:b}}
\end{subfigure}
\caption[Illustrating how operators can be pushed and pulled up and down the tree]{An expression tree representing a query to find which stars worked for which studios in 1996, example taken~\cite[p. 774]{garcia-molina_2002_database_dstcb}. The selection operator in Figure~\ref{fig:pushop:a} is first pulled up the tree, allowing it to then be pushed down an additional branch as illustrated in Figure~\ref{fig:pushop:b}.}\label{fig:pushop}
\end{figure}

\subsubsection{Eliminating operators via an index}
Several operations such as \sql{GROUP BY}, \sql{ORDER BY}, \sql{MAX} etc can be eliminated because the the relation is known to already fulfill these criteria~\cite[p. 777-779]{garcia-molina_2002_database_dstcb}. One of the more common criteria is than index that allows the data to be retrieved sorted exists.

In combination with the ability to move operators up and down the tree this can be very powerful as potentially costly operations can be eliminated.

\subsubsection{Accessing table data}
For each read from a table there will be generated one access path per usable index, as well as one for a full table scan~\cite[p. 827-829]{garcia-molina_2002_database_dstcb}.

\subsection{Cost estimation}
The second task of the query optimizer is to be able to assign a cost to a given search plan. This cost estimation will be repeated several times as it is called for each operator tree in the search space that the query optimizer considers relevant, thus it is important that the estimation is efficient. The cost estimation is done in three steps:
\begin{enumerate}
    \item Collect statistics of stored data;
    \item For each node in the tree calculate the cost of applying the operation;
    \item And then the calculate the statistics for the resulting output.
\end{enumerate}

There are two important statistics that need to be calculated for the data: the number of tuples in a relation and the number of different values in the column of a relation for an attribute~\cite[p. 807-808]{garcia-molina_2002_database_dstcb}. Common methods of saving the data is through a histogram describing the data distribution for an attribute. If histograms are not used the second highest and lowest values are used – this is because the highest and lowest values often are outliers

Modern databases can contain huge amounts of data and thus calculating the exact histograms is impossible, requiring them to be estimated, usually through \textit{sampling}~\cite[p. 807-808]{garcia-molina_2002_database_dstcb}. When estimating via sampling only a subset of the data is read and used to estimate. Furthermore the statistics are usually only computed periodically as they:
\begin{itemize}
    \item Tend not to change radically in a short time;
    \item Even if inaccurate, they are used consistently for all cost estimations;
    \item And keeping them up to date may cause their calculations to take up the most of the database's time.
\end{itemize}

Finally the statistics need to be propagated upwards in the tree. There are several ways of performing this propagation and they are described in more detail in~\cite{chaudhuri_1998_overview_aooqoirs}.

\subsection{Enumeration}
The enumeration is the final task of the optimizer and the one that will perform the actual expansion and cost estimation, as it selects in what way to expand the search space. It would be too costly to expand the entire search space and for each plan estimate the cost, instead the search space is expanded heuristically in a way that the optimizer believes will give cheap plans.

When expanding the search space the general principle is to find paths where the size of relations is reduced as early as possible. For example pushing an operation down the tree, as shown in Figure~\ref{fig:pushop}, can reduce the size of the relation and thus reduce the cost of performing a join later on~\cite[p. 772-774]{garcia-molina_2002_database_dstcb}.

If the enumeration algorithm estimates the cost for a new plan to be more expensive than a previously found one, it can discard it right away. The main goal for the enumeration algorithm is therefore to expand the search space in such a way that the best plans are generated early, so that the more expensive plans can be discarded quickly later on~\cite{nica_2012_analyzing_aqoppojea}.

Most modern query optimizers use a dynamic programming algorithm first proposed in 1979 for the System R database~\cite{selinger_1979_access_apsiardms}. The algorithm is built on the observation that the join methods are independent, that is the best join method for joining the composite to relation $i$ is independent of joining of the first $i-1$ relations. The method used is then to construct a tree with all permutations of joins by searching from smaller to successively larger subsets.

One final optimization done during this step is also to heuristically prune subtrees that the heuristic consider too bad to even consider~\cite{ono_1990_measuring_mtcojeiqo}.

\subsection{Monitoring}
It is often useful to monitor and see what decisions the query optimizer make and why; most databases implement the ability to do so via an SQL statement~\cite[p. 34]{lahdenmaki_2005_relational_rdidatodossea}. In PostgreSQL and MariaDB the statement is called \sql{EXPLAIN}~\cite{postgresql_pd9e}~\cite{explain_emkb}, but it is also called \sql{SHOW PLAN} or \sql{EXPLAIN PLAN}. For an example of how \sql{EXPLAIN} is used see Figure~\ref{fig:sql:explainquery}, which shows the code for a query, and Figure~\ref{fig:sql:explaintrace}, which shows the corresponding trace.

\begin{figure}[ht]
\begin{minted}[breaklines]{sql}
EXPLAIN
SELECT  title.title
FROM    movie_info, title
WHERE   movie_info.info IN ('Bulgaria') AND movie_info.movie_id=title.id;
\end{minted}
\caption[An example query to \sql{EXPLAIN}]{An example of a query done on the IMDb dataset, requesting the title of all movies filmed in Bulgaria. See Figure~\ref{fig:sql:explaintrace} for the output.}\label{fig:sql:explainquery}
\end{figure}

\begin{figure}[ht]
\begin{lstlisting}
 Merge Join  (cost=2.25..921735.28 rows=19682252 width=17)
   Merge Cond: (title.id = movie_info.movie_id)
   ->  Index Scan using title_pkey on title  (cost=0.43..155884.96 rows=3572150 width=21)
   ->  Index Only Scan using movie_info_idx_mid on movie_info  (cost=0.44..511110.22 rows=19682252 width=4)
(4 rows)
\end{lstlisting}
\caption[An example of an EXPLAIN trace]{The access path as shown by PostgreSQL's EXPLAIN statement, corresponding to the the query in Figure~\ref{fig:sql:explainquery}.}\label{fig:sql:explaintrace}
\end{figure}

\subsection{Limitations}
Even though much work has been done on improving query optimizers they may not always choose the correct access path. This section will describe some of the primary reasons why an incorrect access path path is chosen, as described in~\cite[Ch. 14]{lahdenmaki_2005_relational_rdidatodossea}.

\subsubsection{The optimizer can't see the best path}
One reason the query optimizer cannot find the best path is because it is unable to see all alternatives because the query is too complicated for it.

\begin{itemize}
    \item If a predicate is non-indexable it cannot by definition participate in defining the index slice. Furthermore it might also be the case when the predicate is even more difficult that the the optimizer is unable to perform an index screening, forcing it to read a table row.
    \item If a compound predicate contains \sql{OR} it may become non-BT, which in turn mean the predicate cannot be used to define the index slice. This means the query optimizer cannot make full use of potential indexes that exist.
    \item Sometimes the optimizer will add an \sql{ORDER BY} to data that is already sorted thanks to an index.
\end{itemize}

\subsubsection{The optimizer's cost estimate is wrong}
Even if the optimizer is able to see all alternatives it might be the case that the filter factor is incorrectly estimated, resulting in an incorrect access path.

\begin{itemize}
    \item If the filter factor is not estimated for a host variable, it must use a default value which often results in a poor estimate. However, if the filter facotr is estimated evewry time the query is executed it adds a large overhead.
    \item If the optimizer is unaware of the true distribution of the data it might make a guess based on the cardinality, if the distribution is skewed this guess may very well become very incorrect.
    \item In a compound predicate such as \sql{HEIGHT = :HEIGHT AND WEIGHT = :WEIGHT} the optimizer can only produce a good estimate of the filter factor only if it knows the cardinality of the combination of the \sql{HEIGHT} and \sql{WEIGHT} columns. If it does not, it must somehow estimate this.
\end{itemize}

\chapter{Method}\label{chap:method}
    In this chapter, the methods used to investigate the problem statement are
presented. First, the choice of method is described briefly and motivated, this
section will also cover the databases used and why these were picked. Following
this section comes a section describing the benchmark problems, that is the
data set used and the corresponding queries. Finally the implementation details
of the tool used is described in more detail.

\section{Choice of method} \label{sec:choiceofmethod}
This section will cover first why the databases used for evaluation were
chosen, following this dataset used for evaluation is motivated and finally a
short description of the technologies used.

\subsection{Choice of databases} \label{sec:choiceofdatabases}
The two database chosen to be evaluated are PostgreSQL and MariaDB. Both of these databases fulfill the following criteria:
\begin{enumerate}
\item Modern databases that see much use and development;
\item Open-source projects with code that anyone can read, modify and help develop;
\item And they implement state-of-the-art algorithms and methods.
\end{enumerate}

In addition to this PostgreSQL is the typical choice for academic evaluation.
All research papers mentioned in Chapter \ref{chap:relatedwork} that have
implemented new algorithms or modified old ones have done so in PostgreSQL.
MariaDB on the other hand is compatible with MySQL making it a common
alternative for enterprises. The two databases should therefore cover the most
commonly used open-source databases for academia and enterprises.

Studying these two databases should cover how well a modern state-of-the-art
query optimizer performs. In addition, as mentioned in Section \ref{sec:purpose}
since both of them are open-source, if one performs better than the other the
code can be studied to identify areas of improvement.

\subsection{Choice of dataset}
The dataset used for evaluation is one taken directly from the real-world, this
is to ensure that the database has all the traits of a real database:
non-trivial indexes, non-uniform and skewed data as well as a considerable
amount of data.

One important reason that the dataset was selected was that most of the ones
currently used for evaluation such as TPC-H, TPC-DS or more recently JOB (REFERENCE), all
have trivial index setups; that is one index on the primary key for each
relation. As such, studying how the query optimizer selects index for these is
irrelevant: there is just one index trivially found to be correct.

The dataset is based on TriOptima's data, meaning it can't be published. The
relevant statistics of the dataset will be shown in Section \ref{sec:benchmark}

\subsection{Choice of technologies used}
The technologies used to develop the tool used for evaluation were chosen to be
Clojure, using a JDBC driver. The reason these were selected is that Clojure
runs on the JVM, making it platform independent. In addition most of the work
done by the tool is to process and transform data in the form of JSON, a task
that Clojure is well suited to.

\section{Benchmark problems} \label{sec:benchmark}
This section will give a description if the problems used for benchmarking the
query optimizers. The section will start with a description of the relevant
statistics of the data set, following that is a description of the queries
chosen to evaluate the query optimizer.

\subsection{TriOptimas data set}
- size of some tables, number of indexes etc

\subsection{The queries}
- focus on using tables with many indexes
- join several tables to improve errors and increase difficulty for the optimizer
- focus on using realistic queries actually used by trioptima to allow for
better and more realistic modeling

\section{Implementation}
The tool was implemented in Clojure and the evaluation is split up into two
separate processes:
\begin{enumerate}
\item Repeatedly executing queries with different sample sizes;
\item And then parsing the output of the executions to find the access methods used.
\end{enumerate}

The following sections will describe each of the two processes separately.
Finally relevant implementation details, such as how the statistics are deleted,
are provided for each of the two databases.

\subsection{Query execution} \label{sec:queryexecution}
During the query execution data is gathered about one or more queries, repeated
a number of times with a number of different sample sizes used. All resulting
query plans are then saved to a file for later parsing and evaluation.

The execution of one query consists of:
\begin{enumerate}
\item Deleting all statistics gathered by the database;
\item Setting the parameter for what the number of samples used is;
\item Executing an ANALYZE to gather statistics about all tables;
\item Executing an ANALYZE on the query to find what query plan the query
  optimizer selects;
\item And finally saving the resulting JSON to a file;
\end{enumerate}

This procedure is then repeated for each query and each sample size, a number of
times to find all possible query plans used. Pseudocode describing this process
can be found in (REFERERA).

\subsection{Parsing and analyzing} \label{sec:parsing}
During the parsing and analysis a file generated from the previous process
(described in \ref{sec:queryexecution}) is parsed and then analyzed to find what
access methods are used for each relation.

Analyzing one query consists of:
\begin{enumerate}
\item Identifying all relation accesses;
\item Grouping the relation accesses by what relation they access;
\item Finding all unique relation accesses per relation;
\item And finally calculating the size of the unique relation accesses per relation.
\end{enumerate}

This procedure is repeated for each repetition of each query and each sample
size. Pseudocode describing this process can be found in Figure (REFERENCE).

\subsection{PostgreSQL}

\subsection{MariaDB}
\chapter{Results}\label{chap:results}
    This chapter contains the results of it using the tool to evaluate the two
databases. The chapter starts with a section showing the results when
evaluating the effect of the cardinality estimate for the access methods chosen
by the databases. Following this is a section containing the results of the
second evaluation, which focus on evaluating what factors other than cardinality
estimate cause the databases to select different access methods.

\section{The effect of cardinality estimate}\label{sec:correlation}
This section contains the results of the evaluation which focused on the effect of
cardinality estimates. The section is split into two subsections, describing the
results for the first and second query used for evaluation. The options used
when testing the queries is described in Section~\ref{sec:evaluation} and more
specifically Table~\ref{table:evaluation1}.

To illustrate the results found graphs are used to show the number of relations
with varying access methods. This is then plotted over the statistics
targets used to show what effect improving the quality of the cardinality
estimate has on the access methods used. Thus, the x-axis of the graphs shows
the number of relations with varying access methods and the y-axis shows the
statistics target used.

The graphs can be seen as a measure of how consistent the query optimizer is for
the different statistics targets --- the lower the x-value the more consistent
it is.

In addition to the graphs, the full output of the tool is sometimes referred to,
however due to the lengthiness of this it is included in
Appendix~\ref{appendix:output}. The output of the tool is the relevant parts
of the query plans generated, showing the access methods used for each relation.

\subsection{Query \#1}
The first test was done on query \#1, shown in Figure~\ref{fig:sql:query1}, and
the sample sizes used were $1$, $d$ and $2d$ --- capturing the worst scenario, a
reasonable one and a good one. The estimation of cardinality and generation of
query plans was repeated a total of 50 times, as described in
Table~\ref{table:evaluation1}, in order to attempt to capture all access methods
that might reasonably be selected.

The results from this test can be seen in Figure~\ref{fig:plot:eval1:query1}.
The graph shows that PostgreSQL has a total of three relations that are accessed
with different methods, but as the statistics target increases that value
decreases to one. For MariaDB the graph shows that it remains consistent in
always having a varying access method for one relation.

The output of the tool shows that for PostgreSQL the relations \textit{cmm},
\textit{cmm} and \textit{est} are the ones that have varying access methods. The
access methods used are either a full table scan --- a \textit{Seq Scan}
--- an index.

For MariaDB the output shows that one relation remains consistent in being
accessed in multiple different ways --- \textit{ct}. Furthermore, the relation
is accessed using three different indexes. As a matter of fact, no relation is
accessed with a full table scan in MariaDB, it instead always appears to use an
index if one exists on the relation.

\begin{figure}
  \begin{indexplot}
    \addplot coordinates {
      (0,3) (1,1) (2,1)
    };
    \addplot coordinates {
      (0,1) (1,1) (2,1)
    };
  \end{indexplot}
  \caption[The results when evaluating query \#1 with 50 repetitions and a
  varying statistics target.]{The results when evaluating query \#1 with 50
    repetitions and statistics targets of $1$, $d$ and $2d$, where $d$ is the
    default statistics target for the database.}\label{fig:plot:eval1:query1}
\end{figure}

\subsection{Query \#2}
The second test was done on query \#2, shown in Figure~\ref{fig:sql:query2},
which is a subset of the original query involving only five relations. The
tests were once again done with 50 repetitions and a statistics target of $1$, $d$
and $2d$.

The results of the tests can be seen in Figure~\ref{fig:plot:eval1:query2}. The
graph shows the same results as previously observed for PostgreSQL:\@ a bad
quality cardinality estimate will cause it to vary between access methods.
It is this time the relation \textit{mt} which has varying access methods, once
again varying between a full table scan or using an index. As the statistics
target increases PostgreSQL stabilizes and becomes consistent in always opting
to use an index.

For MariaDB the behavior observed in Figure~\ref{fig:plot:eval1:query1} is also
once again observed: the relation \textit{ct} is accessed with three different
indexes.

\begin{figure}
  \begin{indexplot}
    \addplot coordinates {
      (0,1) (1,0) (2,0)
    };
    \addplot coordinates {
      (0,1) (1,1) (2,1)
    };
  \end{indexplot}
  \caption[The results when evaluating query \#2 with 50 repetitions and a
  varying statistics target.]{The results when evaluating query \#2 with 50
    repetitions and statistics targets of $1$, $d$ and $2d$, where $d$ is the
    default statistics target for the database.}\label{fig:plot:eval1:query2}
\end{figure}

\section{Evaluating subsets of the query}\label{sec:subsets}
This section contains the results for the second evaluation conducted. The focus
of this evaluation was to identify what other factors might affect the choice of
access method if it was not the cardinality estimate. Thus, the tests are done
with only 1 repetition to see if the access methods are different even if the
cardinality estimate is the same for all query plans generated.

Three queries are tested, the original query, a subset of the original query
with less tables involved and a trivial query accessing only the relation
\textit{ct}.

The results are presented in the form of bar charts, with each bin of bars
representing a relation and each bar representing the number of different access
methods used for that relation for a database. Thus, a number larger than 1
shows that the relation is accessed using varying access methods.

\subsection{Query \#1}
The first test was done on query \#1 in order to see which access methods
varied, event though only one query plan was retrieved. The results can be
seen in Figure~\ref{fig:plot:eval2:test1}.

The results show that even though the cardinality estimate remains fixed for all
query plans generated, relations are still accessed with multiple access
methods. For MariaDB it is only the relation ct whereas it is relations
\textit{cmt}, \textit{cmm} and \textit{est} for PostgreSQL.

This indicates that there are factors other than the cardinality estimate which
may cause multiple access methods to be used.

\begin{figure}
\begin{indexgraph}
  \addplot coordinates {(ct,3) (t,1) (mt,1) (mm,1) (book,1) (cmt,1) (cmm,1) (est,1) (resamb,1)};
  \addplot coordinates {(ct,1) (t,1) (mt,1) (mm,1) (book,1) (cmt,2) (cmm,2) (est,2) (resamb,1)};
\end{indexgraph}
\caption[The access methods used for query \#1 with 1 repetition.]{The access
  methods when evaluating query \#1 with only 1 repetition. The graph shows that
even though the estimated cardinality is the same for all retrieved query plans,
MariaDB still use different access methods for \textit{ct} and PostgreSQL for
\textit{cmt}, \textit{cmm} and \textit{est}.}\label{fig:plot:eval2:test1}
\end{figure}

\subsection{Query \#2}
The second query tested is simpler than the original as it involves less tables
and thus less \texttt{JOIN} operations. The results of the test can be seen in
Figure~\ref{fig:plot:eval2:test2}, which shows that even though the query is
simpler, PostgreSQL still use multiple access methods relations.

This indicates that the behavior is intentional and not the cause of the query
being sufficiently complex to throw PostgreSQL off and thus cause it to use
varying access methods. Instead, it seems to some extent deliberate.

\begin{figure}
\begin{indexgraph}
  \addplot coordinates {(ct,3) (t,1) (mt,1) (mm,1) (book,1)};
  \addplot coordinates {(ct,1) (t,1) (mt,1) (mm,1) (book,1)};
\end{indexgraph}
\caption[The access methods used for query \#2 with 1 repetition.]{The access
  methods for query \#2 with 1 repetition. The graph shows that even though the
  query is simpler than the original query, PostgreSQL still use different access
  methods for the relation \textit{cmt}.}\label{fig:plot:eval2:test2}
\end{figure}

\subsection{Query \#3}
The final query used used to test the databases with was the most simple variant
of query \#1 --- selecting and filtering only the relation \textit{ct}.
The results of the test can be seen in Figure~\ref{fig:plot:eval2:test3}, which
shows that MariaDB still use three access methods. Furthermore, the output of
the tool shows that the access methods are the same as for all previous tests
with MariaDB.\@

\begin{figure}
\begin{indexgraph}
  \addplot coordinates {(ct,3)};
  \addplot coordinates {(ct,1)};
\end{indexgraph}
\caption[The access methods used for query \#3 with 1 repetition.]{The access
  methods used for query \#3 with 1 repetition. Only one relation is accessed,
  making the query the most simple variant of query \#1 --- yet MariaDB will
  still use three different access methods to access
  \textit{ct}.}\label{fig:plot:eval2:test3}
\end{figure}

\chapter{Discussion}\label{chap:discussion}
    This chapter starts with a discussion regarding the validity of the results and the
possibility of using these to draw more general conclusions. Furthermore the results
and what conclusions can be drawn from these are discussed. Finally, some
suggestions for future research is given.

\section{Validity of the results}\label{sec:validity}
Two main criticisms concerning the validity of the results can be raised:
\begin{enumerate}
\item Only a few number of queries were used for evaluation;
\item And only one dataset was used for evaluation.
\end{enumerate}

Both of these criticisms will be answered in turn below. Finally a motivation
is provided as to why the results can be considered possible to draw general
conclusions from.

\subsection{Only a few queries were used}
\textit{Only one single query and subsets of that query were used, this is hardly a
  large set of tests for evaluation.}

To answer this criticism it is important to consider the fact that the
possibility for multiple access methods for the same relation depends on the
criteria outlined in Section~\ref{sec:dataset}. The tables fulfilling these
criteria are few, reducing the possibility to use multiple queries as they will
only involve the same tables anyway.

Furthermore, the query constructed can be considered to be sufficiently complex
to be problematic for the query optimizer, while not being unreasonably
contrived. This means that if the problem does not arise for the query, it is
highly unlikely to do so for a simpler query --- and of little interest if it
does for a more complex query.

So while only one query and subsets of it are used, they cover a good interval
of simple to reasonably complex for the query optimizer to handle. This means that the
results from these can be considered valid.

\subsection{Only one dataset was used}
\textit{Only one dataset was used for evaluation, the results found for the
  problem studied could be very different for another dataset.}

This criticism is one related to the problem of evaluating databases in general:
the performance of the database is often dependent on the dataset used for
testing. While it is correct that the results could vary depending on dataset,
this problem applies to all studies of databases.

Furthermore, the dataset used to test the databases is one taken from the real
world, making it more realistic than those often otherwise used. As such, the
validity of the results of this study are well on par with those of other
studies using less realistic datasets like TPC-H.

So while only one dataset is used, the results found are realistic and well on
par with that of other studies on the subject of databases.

\subsection{Applicability}
The results found in this study do, as all other studies of databases do, suffer
from the problem of the possibility of the results depending on the dataset
used. However, the results found in this study show two things:
\begin{itemize}
\item Different access methods are used to access the same relation;
\item And the behavior of when this is done differs between MariaDB and
  PostgreSQL.\@
\end{itemize}

Neither of these results is tied to the specific dataset used, instead they are
both show behavior that will very likely arise for more datasets. Adding more
datasets could only answer the question of the prevalence of the behavior.

\section{Selection of access method}\label{sec:accessmethods}
This section will discuss the main findings of the evaluation, starting with a
discussion regarding the correlation between sample size and the number of
different access methods used for the same relation. Following this a section
describing the second set of results identified: the correlation between
predicate value and access methods used.

\subsection{The effect of cardinality estimate}
The main purpose of the study was to identify the effect of the cardinality
estimate on the access methods selected. As a way to evaluate this the number of
samples used when analyzing and estimating the cardinality is varied from a low
to a high value. The results found from the evaluation show the following:
\begin{enumerate}
\item PostgreSQL will vary between doing a full table scan or accessing the data
  via an index depending on the cardinality estimate;
\item PostgreSQL becomes more consistent in using the same access method when
  the cardinality is better estimated thanks to an increased sample size;
\item The cardinality estimate rather than the number of joins seem to be the
  main cause of PostgreSQL varying access method;
\item MariaDB remains unaffected by the cardinality estimate for the tests done;
\item The primary reason why MariaDB remains unaffected seems to be the fact
  that it will always use an index if one exists, never doing a full table scan.
\item And finally, MariaDB and PostgreSQL will sometimes use different access
  methods, even for a cardinality estimate of high quality.
\end{enumerate}

\subsubsection{PostgreSQL varies between full table scan and using an index}
That PostgreSQL varies between full table scans and using an index depending on
the cardinality estimated can be seen in Figure~\ref{fig:plot:eval1:test1} where
the relations \textit{mt}, \textit{cmt}, \textit{cmm} and \textit{est} are all
accessed with two different methods. The output of the tool, seen in
Figure~\ref{fig:json:eval1:test1:postgresql}, shows that these accesses are
either using an index or a full table scan.

It is also the case that this does not happen because of the varying predicate
value, as the relation \textit{mt} has varying access methods --- which is not
the case when testing with only one repetition, as can be seen in
Figure~\ref{fig:plot:eval2:test1}.

\subsubsection{PostgreSQL is more consistent for better cardinality estimates}
When testing PostgreSQL with a better cardinality estimate, as shown in
Figure~\ref{fig:plot:eval1:test3}, the number of relations which have different
access methods decrease. With the high sample size it is only the relation
\textit{cmm} that is accessed in multiple ways.

As can be seen in Figure~\ref{fig:json:eval1:test3:postgresql}, PostgreSQL will also choose
to use an index when the cardinality is better estimated. All of the relations
but \textit{cmm} are now accessed only using an index. This indicates that the
behavior PostgreSQL's query optimizer considers correct is using an index for
these relations, but when the cardinality estimate varies due to a low sample
size it will sometimes opt for an incorrect full table scan.

\subsubsection{Cardinality estimate is the reason rather than complexity caused
  by joins}
The tests done with the simpler query, query \#2, showed that even though the
query is simpler in terms of the number of joins and tables involved, PostgreSQL
will still vary in what access methods its using, as can be seen in
Figure~\ref{fig:plot:eval1:test2}. This can be seen as a further motivation that
the cardinality estimate, rather than other factors, is what is causing
PostgreSQL to vary between what access methods its using.

\subsubsection{MariaDB is unaffected by cardinality estimate}
Unlike PostgreSQL, MariaDB seems to remain unaffected by the cardinality
estimates as can be seen when comparing the results in
Figure~\ref{fig:plot:eval1:test1} to those in Figure~\ref{fig:plot:eval1:test3}.
It is in both only one relation, \textit{ct}, that is accessed with multiple different
access methods and that relation has the same number of different access methods
--- three --- for both tests. It is also the same access methods that are used, as
can be seen in Figure~\ref{fig:json:eval1:test1:mariadb} and
Figure~\ref{fig:json:eval1:test3:mariadb}.

\subsubsection{MariaDB always use an index if one exists}
Another difference to PostgreSQL is that MariaDB will always use an index if one
exists. This can be seen in Appendix~\ref{appendix:output} where the full output
of the tool is shown. For all relations which have only two possible access
methods --- via an index on the primary key or through a full table scan ---
MariaDB will only ever consider using the index. It is therefore only the case
that relations \textit{ct} and \textit{mt} can have several access methods as
they have more than one index.

\subsubsection{Both databases use many access methods for high quality
  cardinality estimates}
The final discovery that can be drawn from the results is that both databases
use multiple access methods even for high quality estimates of the cardinality.
As an example consider Figure~\ref{fig:plot:eval1:test3}, the relation
\textit{ct} for MariaDB and \textit{cm}m for PostgreSQL are both accessed with
multiple access methods even though a high sample size is used when estimating
the cardinality.

Furthermore, as MariaDB remains consistent in choosing between the three
different methods regardless of the sample size used when estimating cardinality
it seems that the choice is a deliberate one. This behavior was therefore
further evaluated by using only 1 repetition to see if the reason for varying
behavior depended on another factor than varying cardinality estimates. The
results for this evaluation will be discussed in the next section,
Section~\ref{sec:predicatecorrelation}.

\subsection{The effect of predicate value}\label{sec:predicatecorrelation}
The second evaluation done was focused on identifying if a factor other than
cardinality estimate was the reason for the varying use of access methods. This
evaluation was therefore done by using only 1 repetition and so on only one
cardinality estimate is the result for all the query plans generated.

The evaluation found the following results:
\begin{enumerate}
\item Both databases will use different access methods for the same estimated
  cardinality;
\item The factor that is the reason for the different access methods used is the
  predicate used to filter rows;
\item The behavior is deliberate and will remain regardless of query complexity.
\end{enumerate}

\subsubsection{Both databases use different access methods}
The first notable result is that both databases will use different access
methods even when the estimated cardinality is the same. This can be clearly
seen in Figure~\ref{fig:plot:eval2:test1} where the relation \textit{ct} for
MariaDB and relations \textit{cmt}, \textit{cmm} and \textit{est} are all
accessed with multiple different access methods.

\subsubsection{The predicate value used is important}
The reason that these relations are accessed in different ways can not be
because of a varying cardinality estimate, as the same estimated cardinality is
used when generating all query plans. The only other thing that varies in the
tests is that all possible values for \sql{:KEY} in the query, seen in
Figure~\ref{fig:sql:query1}, are used to generate query plans. From this it can
be concluded that the the value of \sql{:KEY} in the predicate \sql{ct.key =
  :KEY} will affect the access method for the relations.

It can be further noted that for PostgreSQL the predicate will affect relations
that are not directly involved in it, but also those that are joined together
with it (for example \textit{cmt}). Since MariaDB always use an index if one
exists this behavior is hard to observe as the only relation it will use
different access methods for is the one in the predicate, \textit{ct}.

\subsubsection{The behavior remains for simpler queries.}
Finally, the behavior of being sensitive to predicate value was evaluated for
two simpler queries to see if the result remains the same when not
\texttt{JOIN}ing so many tables together.

The first test was done on query \#2 and the results can be seen in
Figure~\ref{fig:plot:eval2:test2}. The main focus was on evaluating PostgreSQL
to see if it behaved the same and varied access method for the relation
\textit{cmt}. As can be seen in the figure and when comparing the output, shown
in Figure~\ref{fig:json:eval2:test1:postgresql} and
Figure~\ref{fig:json:eval2:test2:postgresql} respectively, PostgreSQL does
continue to use different access methods.

For MariaDB the behavior was tested using query \#3, which is a simple select
only on the relation \textit{ct}. The results can be seen in
Figure~\ref{fig:plot:eval2:test3} and it is clear that the behavior remains ---
three different access methods are used. Comparing the output, seen in
Figure~\ref{fig:json:eval2:test1:mariadb} and
Figure~\ref{fig:json:eval2:test3:mariadb}, also shows that the methods are the
same.

\subsection{Conclusions from the discussion}
This section will discuss what conclusions can be drawn from the discussions
regarding the effect of cardinality and predicate value.

First of all, cardinality estimate has a clear effect on PostgreSQL and the
access methods chosen by its query optimizer. When the sample size used to
estimate cardinality is low the number of different access methods used for
relations will vary much more than when the sample size used increases. This
result provides a good indication as to the question that this thesis asks, and
shows that cardinality estimate will affect access method chosen for PostgreSQL.\@

The effect of cardinality estimate can however only be observed for PostgreSQL,
for MariaDB no such correlation can be seen. The primary reason that MariaDB is
not as sensitive to cardinality estimate is that it never considers doing a full
table scan if it can use an index. However, it is worth noting that it can not
be concluded that MariaDB is unaffected by cardinality estimates. Instead it
might be concluded that it is less likely for MariaDB to vary between different
access methods because of different cardinality estimates, as it must be the
case that it is different indexes it varies between --- rather than an index or
a full table scan, as is the case for PostgreSQL.\@

When evaluating the databases it is also found that another factor that affects
the choice of access method is the predicate value used to filter rows. For
PostgreSQL till affects whether it uses an index or does a full table scan,
whereas it for MariaDB means it will switch between using different indexes.
Since it is usually the case that one index should be the correct one for a
given query this behavior might be seen as unintuitive. Furthermore, this
behavior might cause a query that was first fast to become fast with the
Introduction of a seemingly unrelated index, a story of this happening to a
company is described in~\cite{lahdenmaki_2005_relational_rdidatodossea}[Ch.~14].

One final result that is noteworthy is that the two query optimizers behave
considerably different for the tests done. PostgreSQL never considers
many different indexes for the same relation, unlike MariaDB.\@ MariaDB on the
other hand never considers a full table scan, unlike PostgreSQL.\@ This is
interesting as it shows that there seems to be no clear best practice for query
optimizers if they differ on such fundamental levels. This also indicates that
further research and evaluation of query optimizer is necessary in order to find
what is the best behavior.

\section{Future research}
This section will cover some suggestions for future research on the topic of
databases, both in general and specifically for the problem of selecting access method.

As discussed in Section~\ref{sec:validity} one problem when evaluating
databases is the dataset used for evaluation. In this study a dataset based on a
product for the company TriOptima was used to evaluate the databases with a real-world
dataset. However, this dataset can't be made public. Other datasets, like TPC-H
or the more recent JOB, suffer from the problem that they are simpler
than a database used by a company; as an example both of them have only one
index per relation on the primary key.

One important area of research in databases would therefore be to create one or
more realistic datasets with complex data, relations and indexes that could be
used for research. Using these datasets for evaluation would then provide
results that could be considered more general and correct.

The focus of this thesis was the effect of the cardinality estimate on the
access methods used by the query optimizer. The results show that PostgreSQL is
more sensitive to this and will more often do a full table scan when the
cardinality estimate is done with a lower sample size. MariaDB on the other hand
is observed to be more robust as it never does a full table scan if an index
exists. This leads to the question of which behavior is the best --- should the
query optimizer allow full table scans or is it better to always skip them?

This study also show that both database's query optimizers select
different access methods depending on predicate value and that the two do so
differently. This leads to the question of whether this is the correct behavior
or if the query optimizers should try to use only one access method always,
regardless of predicate value.

Finally it can be noted that the results from this study show the need for
further research into query optimizers as two state-of-the-art query optimizers
behave considerably different. The topic of query optimization clearly needs
much further study to arrive at what is best practices.

\chapter{Conclusions}\label{chap:conclusions}
    This thesis concerns the evaluation of two modern, state-of-the-art database's,
and the effect of estimating statistics on the index selection. The thesis question was:
\textit{How much effect does the cost estimation have on the query optimizers selection of indexes during the join enumeration?}

Two databases --- PostgreSQL and MariaDB --- were chosen as a good representation of
state-of-the-art query optimizers. A large enterprise dataset was ported to both
of these two databases. Finally, a query was constructed using the most complex
tables in the dataset.

A tool was then implemented in Clojure to allow the two databases to be
evaluated. With the tool the index selections for a given query can be tested
for a number of different sample sizes. By testing over an increasing sample
size the cost estimation can be simulated to range from bad to good, allowing
the testing of databases in a realistic scenario.

Using the tool to evaluate the databases with the query we find that the cost
estimation have little effect on the index selection. For all the queries used,
regardless of sample size, no correlation is found between the sample size and
the number of different indexes used to access the same relation.

Instead the results indicate that predicate values can cause the databases to
select different indexes. Furthermore, it is found that this behavior is
different for the two databases with neither being sensitive to the same value
as the other. This shows that the query optimizer's perform different for the
two databases, indicating that further study should be done to identify which of
the two behaviors might be considered best.

In conclusion we have found results indicating that the cost estimations have
little effect on the query optimizers selection of access method. We have also found
results indicating that instead the predicate value used for filtering does
effect the index selection.
\nocite{*}
\printbibliography{}

\appendix
\chapter{Program output}\label{appendix:output}
\section{Evaluation 1}

\subsection{PostgreSQL}
\subsubsection{Query \#1}
\json{json/postgresql1-1.json}{PostgreSQL}{\#1}{1}{50}
\json{json/postgresql1-d.json}{PostgreSQL}{\#1}{d}{50}
\json{json/postgresql1-2d.json}{PostgreSQL}{\#1}{2d}{50}

\subsubsection{Query \#2}
\json{json/postgresql4-1.json}{PostgreSQL}{\#1}{1}{50}
\json{json/postgresql4-d.json}{PostgreSQL}{\#1}{d}{50}
\json{json/postgresql4-2d.json}{PostgreSQL}{\#1}{2d}{50}

\subsection{MariaDB}
\subsubsection{Query \#1}
\json{json/mariadb1-1.json}{MariaDB}{\#1}{1}{50}
\json{json/mariadb1-d.json}{MariaDB}{\#1}{d}{50}
\json{json/mariadb1-2d.json}{MariaDB}{\#1}{2d}{50}

\subsubsection{Query \#2}
\json{json/mariadb4-1.json}{MariaDB}{\#1}{1}{50}
\json{json/mariadb4-d.json}{MariaDB}{\#1}{d}{50}
\json{json/mariadb1-2d.json}{MariaDB}{\#1}{2d}{50}

\section{Evaluation 2}
\subsection{PostgreSQL}
\json{json/postgresql1-1-1.json}{PostgreSQL}{\#1}{1}{1}
\json{json/postgresql3-1-1.json}{PostgreSQL}{\#2}{1}{1}
\json{json/postgresql9-1-1.json}{PostgreSQL}{\#3}{1}{1}

\subsection{MariaDB}
\json{json/mariadb1-1-1.json}{MariaDB}{\#1}{1}{1}
\json{json/mariadb3-1-1.json}{MariaDB}{\#2}{1}{1}
\json{json/mariadb9-1-1.json}{MariaDB}{\#3}{1}{1}
\end{document}
