In this chapter we discuss the results, the validity of these in more general terms. Finally some ideas for future research are presented.

\section{Validity}\label{sec:validity}
Two main criticisms concerning the validity of the results can be raised:
\begin{enumerate}
\item Only a few number of queries were used for evaluation;
\item And only one dataset was used for evaluation.
\end{enumerate}

Both of these criticisms will be answered in turn below. Following this an
additional motivation as to the validity of the results is provided.

\subsection{Only a few queries were used}
\textit{Only one single query and subsets of that query were used, this is hardly a
large set of tests for evaluation.}

To answer this criticism its important to
consider the important fact that the possibility for ambiguous indexes depends
on the criteria outlined in Section~\ref{sec:dataset}. The tables fulfilling
these queries are few, reducing the possibility to use multiple queries as they
will only involve the same tables anyway.

Furthermore, the query constructed can be considered to be sufficiently complex
as to capture a problematic query that the problem would realistically arise
for. From this it can be generalized as to say that if the problem does not
arise for this query, it is highly unlikely that any simpler query would cause
problems and more complex queries would be unnecessary contrived.

So while only one query and subsets of it are used, these queries cover a good
range of queries from simple to more complex for the query optimizer. This means
that the results from these can be considered valid.

\subsection{Only one dataset was used}
\textit{Only one dataset was used for evaluation, the results found for the
  problem studied could be very different for another dataset.}

This criticism is one bound to the problem of evaluating databases in general:
the performance of the database is often dependent on the dataset used for
testing. While it is correct that the results could vary depending on dataset,
any study of databases will suffer from this problem and the results should be
considered an indication more than anything else.

Furthermore, the dataset used to test the databases is one taken from the real
world, making it more realistic than those often otherwise used. As such, the
validity of the results of this study are well on par with those of other
studies using less realistic datasets like TPC-H.

While only one dataset is used for evaluation, the results found can be seen as
an indication of a specific behavior and should warrant future research.
Additionally, the results found are based on a dataset taken from the real
world further giving validity to the results.

\subsection{Further motivation}
Studying databases is problematic because it is out of necessity bound to the
underlying dataset. This problem is unavoidable and can only be limited by the
creation of more datasets for testing.

Maybe the most important motivation to the validity of the results is the fact
that the behavior seen is present. How common the problem of incorrect index
selection is is hard to say, but that it does exist the results seem to
indicate. Therefore the results can be seen as an indication of an existing
problem, but not an indication of how common they are.

\section{Ambiguous indexes}
- index selections tend to not be ambiguous, the optimizer remains consistent regardless of sample size used
- ambiguous indexes in terms of parameter values are however present

\section{Future research}
This section will cover some suggestions for future research on the topic of
databases, both in general and specifically for the problem of index selection.

As discussed in Section~\ref{sec:validity} one problem when evaluating
databases is the dataset used for evaluation. In this study a dataset based on a
product for the company TriOptima was used to provide an example of a real-world
dataset. However, this dataset can't be made public.

Other datasets like TPC-H or to more recently create JOB suffer from the problem
that they are simpler than a database used by a company; as an example both of
them have only one index per relation on the primary key.

One important area of research in databases would therefore be to create one or
more realistic datasets with complex data, relations and indexes that could be
used for research. Using these datasets for evaluation would then provide
results more general and correct.

The most important results indicated by this study is that query optimizer
selects different indexes depending on the parameter value for queries. If this
is the correct choice or not to do is not evaluated as a performance study is
beyond the scope of this thesis. A study, evaluating the performance of the
different indexes used should be conducted to see if being sensitive to query
parameters is correct.

Finally it is important to also note that the problem of ambiguous indexes
because of small sample size, or incorrect estimation of statistics, seem to not
be common. As such, future study into the area should probably avoid this topic
in favor of other topics, for example that of optimizers being sensitive to
parameter values.