This master thesis concern relational databases and their query optimizer's
sensitivity to cardinality estimates and the effect the quality of the estimate
has on the number of different access methods used for the same relation. Two
databases are evaluated --- PostgreSQL and MariaDB --- on a real-world dataset
to provide realistic results. The evaluation was done via a tool implemented in
Clojure and tests were conducted on a query and subsets of it with varying
sample sizes used when estimating cardinality.

The results indicate that MariaDB's query optimizer is less sensitive to
cardinality estimates and for all tests select the same access methods,
regardless of the quality of the cardinality estimate. This stands in contrast
to PostgreSQL's query optimizer which will vary between using an index or doing
a full table scan depending on the estimated cardinality. Finally, it is also
found that the predicate value used in the query affects the access method used.
Both PostgreSQL and MariaDB are found sensitive to this property, with MariaDB
having the largest number of different access methods used depending on
predicate value.
