The terminology of databases often varies across literature and database
vendors, this section therefore defines the terms used in this thesis. When
it is the case that something commonly goes by other names as well, an attempt
has been made to include these names as well.

\subsection*{Database}
No distinction is made between the database and the Database Management Systems
(DBMS) in this thesis and the terms are used interchangeably.

\subsection*{Data}
The \textit{data} in a database is the values stored in the rows and columns of
the database. This does not include additional information stored in the
database such as indexes.

\subsection*{Dataset}
A \textit{dataset} is all information stored in the database, including both
data and additional information such as indexes, primary and foreign keys etc.

\subsection*{Query optimizer}
The terms query optimizer and optimizer are used interchangeably throughout the
thesis. If an optimizer of some other kind is used this will be made explicit.

\subsection*{Host variable}
A \textit{host variable} is a variable declared in the program in which the SQL
statement is embedded~\cite[p. 151]{chamberlin_1998_complete_acgtdud}. Host
variables are identified by the fact that they begin with a colon. An example of
a host variable is \sql{:HEIGHT} in Figure~\ref{fig:sql:hostvar}.

\begin{figure}[ht]
  \begin{minted}[breaklines]{sql}
    SELECT  NAME
    FROM    PERSONS
    WHERE   HEIGHT = :HEIGHT
  \end{minted}
  \caption[A query with a host variable]{A simple query using a host
    variable.}\label{fig:sql:hostvar}
\end{figure}

\subsection*{Predicate}
In order to get the data you need, you must be able to specify what conditions
the data should fulfill to be relevant, this is done by specifying predicates.
To illustrate, consider Figure~\ref{fig:sql:predicate} (example taken
from~\cite{lahdenmaki_2005_relational_rdidatodossea}).

\begin{figure}[ht]
  \begin{minted}[breaklines]{sql}
    WHERE   SEX = 'M'
    AND
    (WEIGHT = 90
    OR
    HEIGHT > 190)
  \end{minted}
  \caption[The \sql{WHERE} clause of a query containing three predicates and two
  compound predicates]{The \sql{WHERE} clause an SQL query containing three
    predicates and two compound predicates.}\label{fig:sql:predicate}
\end{figure}

The \sql{WHERE} clause in Figure~\ref{fig:sql:predicate} contains three
predicates:
\begin{enumerate}
\item \sql{SEX = 'M'}
\item \sql{WEIGHT = 90}
\item \sql{HEIGHT > 190}
\end{enumerate}

A \textit{compound predicate} is two or more predicates that are tied together
in the form of \sql{AND}, \sql{OR} or other similar operators. The \sql{WHERE}
clause in Figure~\ref{fig:sql:predicate} can be considered to have two different
compound predicates:
\begin{enumerate}
\item \sql{WEIGHT = 90 OR HEIGHT > 190}
\item \sql{SEX = 'M' AND (WEIGHT = 90 OR HEIGHT > 190)}
\end{enumerate}

\subsection*{Index slice}
The term \textit{index slice} comes
from~\cite{lahdenmaki_2005_relational_rdidatodossea} and is defined as the
number of index rows that need to be read for a predicate; the thinner the slice
the less amount of index rows that need to be read, and consequently the number
of reads to the table.

The thickness of the index slice will depend on the number of \textit{matching
  columns} --- the number of columns that exist both in the \sql{WHERE}
clause and the index. To illustrate why, consider the query in
Figure~\ref{fig:sql:indexslice}.

\begin{figure}[ht]
  \begin{minted}[breaklines]{sql}
    WHERE   WEIGHT = 90
    AND
    HEIGHT > 190
  \end{minted}
  \caption[The \sql{WHERE} clause of a query with two potential matching
  columns]{The \sql{WHERE} clause of a query with two potential matching
    columns.}\label{fig:sql:indexslice}
\end{figure}

If there exists an index on only \sql{HEIGHT}, no values for \sql{WEIGHT} can be
discarded in the index slice. If an index is added for \sql{WEIGHT}, the
thickness of the index slice will decrease as only values fulfilling both the
\sql{HEIGHT} and \sql{WEIGHT} requirements remain.

\subsection*{Indexable predicate}
A \textit{indexable predicate} is a predicate that can be evaluated when the
index is accessed (allowing a matching index scan)~\cite{2014_summary_sopp,
  2013_ibm_ikcianp}. Revisiting the example from earlier, both of the predicates
in Figure~\ref{fig:sql:indexslice} are examples of indexable predicates.

\subsection*{Matching predicate}
A \textit{matching predicate} is an indexable predicate with the corresponding
necessary indexes~\cite{2013_ibm_ikcianp}. In Figure~\ref{fig:sql:indexslice}
both predicates are indexable and would be matching if there exists an index for
\sql{WEIGHT} and \sql{HEIGHT} respectively.

\subsection*{Non-indexable predicate}
A \textit{difficult predicate} (also sometimes called \textit{nonsearch
  arguments}, \textit{index suppression}, \textit{difficult predicate}) is the
opposite of an indexable predicate, and can as a consequence not define the
index slice~\cite{lahdenmaki_2005_relational_rdidatodossea}. What predicates are
non-indexable varies from database to database, but a typical example of one can
be seen in Figure~\ref{fig:sql:nonindexable}.

\begin{figure}[ht]
  \begin{minted}[breaklines]{sql}
    COL1 NOT BETWEEN :hv1 AND :hv2
  \end{minted}
  \caption[An example of a non-indexable predicate]{A example of a commonly used
    non-indexable predicate.}\label{fig:sql:nonindexable}
\end{figure}

\subsection*{Boolean term predicate}
A \textit{boolean term predicate} (BT predicate) is one that can reject a row
because it does not match the
predicate~\cite{lahdenmaki_2005_relational_rdidatodossea}. Conversely a non-BT
predicate is a predicate that cannot reject a row. Non-BT predicates are
typically the result of using \sql{OR}. To illustrate when a predicate is BT
respectively non-BT consider, assume there is an index \texttt{(A, B)} on
\sql{TABLE} and consider Figure~\ref{fig:sql:btpredicate} and
Figure~\ref{fig:sql:nonbtpredicate}.

\begin{figure}[ht]
  \begin{minted}[breaklines]{sql}
    SELECT  A, B
    FROM    ATABLE
    WHERE   A > :A
    AND
    B > :B
  \end{minted}
  \caption[A query containing a BT predicate]{A query with a BT
    predicate.}\label{fig:sql:btpredicate}
\end{figure}

\begin{figure}[ht]
  \begin{minted}[breaklines]{sql}
    SELECT  A, B
    FROM    ATABLE
    WHERE   A > :A
    OR
    B > :B
  \end{minted}
  \caption[A query containing no BT predicates]{An query with no BT
    predicates.}\label{fig:sql:nonbtpredicate}
\end{figure}

For the query in Figure~\ref{fig:sql:btpredicate} if the first predicate \sql{A
  > :A} evaluates to false for a row the row can be rejected instantly, making
it a BT predicate. For the query in Figure~\ref{fig:sql:nonbtpredicate} on the
other hand it might be the case that \sql{B > :B} evaluates to true even if
\sql{A > :A} does not, making both predicates non-BT predicates.

\subsection*{Index screening}
A column may be in both the \sql{WHERE} clause and the index, yet be unable to
participate in defining the index slice due to other
reasons~\cite{lahdenmaki_2005_relational_rdidatodossea}. Even if this is the
case the column may still be able to reduce the amount of reads to the table
anyway. A column fulfilling these criteria is a \textit{screening column} as the
presence of it in the index allows not reading from the table. The process of
determining which columns might fulfill this is called \textit{index screening}.

\subsection*{Cardinality}
The \textit{cardinality} is the number of distinct values for a column, or
combination of columns~\cite{lahdenmaki_2005_relational_rdidatodossea}. The
cardinality of the data is usually used when the query optimizer estimates the
cost of different access paths.

\subsection*{Statistics target}
The \textit{statistics target} is used to refer to the amount of statistics
stored when analyzing a table, and thus by extension the sample size used when
estimating the cardinality of the relations. What the statistics target is
varies between database implementations.

\subsection*{Filter factor}
The \textit{filter factor} specifies what proportion of the rows that    satisfy
the conditions in a predicate~\cite{lahdenmaki_2005_relational_rdidatodossea}.
The filter factor can be seen as the selectivity of a predicate and the lower it
is, the more the number of rows that are filtered out by a predicate. For a
predicate such as \sql{HEIGHT = :HEIGHT} there are three ways to talk about
filter factor:
\begin{itemize}
\item The \textit{value specific filter factor} is the filter factor for one
  specific value of \sql{:HEIGHT};
\item The \textit{average filter factor} is the average value for all value
  specific filter factors;
\item And the \textit{worst-case filter factor} is the highest possible filter
  factor for a given value of \sql{:HEIGHT}
\end{itemize}

\subsection*{Access path}
The query optimizers output is an \textit{access path}, which is an abstract
representation of the path to access the data.

\subsection*{Query plan}
The \textit{query plan} will be used to refer to the concrete representation
given by an database to describe the underlying access path.

\subsection*{Execution plan}
The \textit{execution plan} corresponds to an access path but describes how to
physically access the data.

\subsection*{Relation}
The words \textit{relation} and \textit{table} are synonymous and used
interchangeably throughout the thesis.