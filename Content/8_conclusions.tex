This thesis concerns the evaluation of two modern, state-of-the-art database's,
and the effect of cardinality estimates on the access methods used; the thesis
question posed was:
\textit{How much effect does the cardinality estimate have on the query optimizers
  selection of access method during the join enumeration?}

Two databases --- PostgreSQL and MariaDB --- were chosen as good representations of
state-of-the-art query optimizers. A large enterprise dataset was ported to both
of these two databases. Finally, a query was constructed using the most complex
tables in the dataset.

A tool was then implemented in Clojure to allow the two databases to be
evaluated. With the tool the access methods used for each relation can be
identified and tests conducted with different queries, sample sizes and
repetitions.

An evaluation of the two databases was done using the query constructed and
testing their choice of access methods for varying sample sizes. The results
from this evaluation warranted a second evaluation using the same estimated
cardinality to identify other factors than cardinality estimate that might cause
the databases to select different access methods.

The results show that PostgreSQL is affected by bad cardinality estimates, and
will vary between using an index or doing a full table scan for more relations
when the sample size is kept small. MariaDB on the other hand was found to be
less sensitive to cardinality estimate, mainly because it would never consider
doing a full table scan if it could instead use an index.

The second evaluation found that MariaDB will instead vary between using
different indexes depending on the predicate value used when filtering rows.
This was found to be true regardless of the complexity of the query and the
behavior remained consistent even for a trivial query involving only one table.
PostgreSQL also displayed a similar behavior, but instead switched between using
an index or doing a full table scan depending on predicate value.

In conclusion, we have found results showing that the cardinality estimate
affects PostgreSQL and smaller samples used when estimating will increase the
risk of doing a full table scan instead of using the correct index. MariaDB was
found to be more robust to varying cardinality estimates, because it will always
use an index if one exists.