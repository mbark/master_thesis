This thesis concerns the evaluation of two modern, state-of-the-art database's,
and the effect of cardinality estimates on the access methods used; the thesis
question posed was:
\textit{How much effect does the cardinality estimate have on the query optimizers
            selection of access method during the join enumeration?}

Two databases --- PostgreSQL and MariaDB --- were chosen as good representations of
state-of-the-art query optimizers. A large enterprise dataset was ported to both
of these two databases. Finally, a query using the most complex
tables in the dataset was constructed.

A tool was then implemented in Clojure to allow the two databases to be
evaluated. With the tool the access methods used for each relation can be
identified and tests conducted with different queries, statistics targets and
repetitions.

An evaluation of the two databases was done using the query constructed and
testing their choice of access methods for varying statistics targets. The results
from this evaluation warranted a second evaluation to identify other factors
than cardinality estimate that might cause the databases to select different
access methods.

The results show that PostgreSQL is clearly affected by bad cardinality
estimates with several relations being accessed using multiple different access
methods. It is furthermore found that for all these relations PostgreSQL will,
due to bad cardinality estimates, incorrectly opt for a full table scan instead
of using an index. With an increased statistics target used for the cardinality
estimate PostgreSQL is found to become more consistent and the number of
relations with varying access methods decrease --- but does not reach zero.

MariaDB is found to be unaffected by bad cardinality estimates and remains
consistent in what query plans are generated, regardless of statistics target
used. The main reason identified for this is that MariaDB will always use an
index if one exists, thus making it impossible to do the same mistake as
PostgreSQL does. However, one relation is consistently selected using multiple
access methods.

The reason that the databases used varying access methods for the same data even
with a high statistics target was studied in the second evaluation. The results
found that both were sensitive to the predicate value used when filtering the
rows selected. MariaDB would vary between using three different predicates
depending on what predicate value was used. This behavior was also observed for
PostgreSQL, except it would vary between using an index or doing a full table scan.

In conclusion, we have found results showing that for PostgreSQL the cardinality
estimate clearly affects what access methods are used --- with bad estimates
resulting in incorrect full table scans. MariaDB was found to not be affected
by cardinality estimate, mainly because it would always use an index if one existed.