This thesis concerns the evaluation of two modern, state-of-the-art database's,
and the effect of estimating statistics on the index selection. The thesis
question posed was:
\textit{How much effect does the cost estimation have on the query optimizers
  selection of access method during the join enumeration?}

Two databases --- PostgreSQL and MariaDB --- were chosen as a good representation of
state-of-the-art query optimizers. A large enterprise dataset was ported to both
of these two databases. Finally, a query was constructed using the most complex
tables in the dataset.

A tool was then implemented in Clojure to allow the two databases to be
evaluated. With the tool the access methods used for each relation can be
identified and tests conducted with different queries, sample sizes and repetitions.
By testing over an increasing sample size the cost estimation can be simulated
to range from bad to good, allowing the testing of databases in a realistic
scenario.

Using the tool to evaluate the databases with the query we find that the
cardinality estimates have little effect on the access methods chosen by
MariaDB, whereas PostgreSQL will vary between accessing the data via a full
table scan or an index depending on the estimated cardinality.

The results also indicate that predicate values cause the query optimizers to
select different access methods. Furthermore, it is found that this behavior is
different for the two databases.

In conclusion we have found results indicating that the cardinality estimate has
little effect on MariaDB, whereas PostgreSQL is more sensitive to cardinality
estimates. We have also found results indicating that instead the predicate
value used for filtering does effect the index selection.