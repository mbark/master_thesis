This thesis concerns the evaluation of two modern, state-of-the-art database's,
and the effect of estimating statistics on the index selection. The thesis question was:
\textit{How much effect does the cost estimation have on the query optimizers selection of indexes during the join enumeration?}

Two databases --- PostgreSQL and MariaDB --- were chosen as a good representation of
state-of-the-art query optimizers. A large enterprise dataset was ported to both
of these two databases. Finally, a query was constructed using the most complex
tables in the dataset.

A tool was then implemented in Clojure to allow the two databases to be
evaluated. With the tool the index selections for a given query can be tested
for a number of different sample sizes. By testing over an increasing sample
size the cost estimation can be simulated to range from bad to good, allowing
the testing of databases in a realistic scenario.

Using the tool to evaluate the databases with the query we find that the cost
estimation have little effect on the index selection. For all the queries used,
regardless of sample size, no correlation is found between the sample size and
the number of different indexes used to access the same relation.

Instead the results indicate that predicate values can cause the databases to
select different indexes. Furthermore, it is found that this behavior is
different for the two databases with neither being sensitive to the same value
as the other. This shows that the query optimizer's perform different for the
two databases, indicating that further study should be done to identify which of
the two behaviors might be considered best.

In conclusion we have found results indicating that the cost estimations have
little effect on the query optimizers selection of access method. We have also found
results indicating that instead the parameter values used for filtering does
effect the index selection.