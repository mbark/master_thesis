Detta examensarbete behandlar relationella databaseer och hur stor
påverkan kvaliteten på den uppskattade kardinaliteten har på antalet olika
metoder som används för att hämta data från samma relation. Två databaser
testades --- PostgreSQL och MariaDB --- på ett verkligt dataset för att ge
realistiska resultat. Utvärderingen gjordes med hjälp av ett verktyg
implementerat i Clojure och testerna gjordes på en query, och delvarianter av
den, med varierande stora sample sizes för kardinalitetsuppskattningen.

Resultaten indikerar att MariaDBs query optimizer inte påverkas av
kardinalitetsuppskattningen, för alla testerna valde den samma metod för att
hämta datan. Detta skiljer sig mot PostgreSQLs query optimizer som varierade
mellan att använda sig av index eller göra en full table scan beroende på den
uppskattade kardinaliteten. Slutligen pekade även resultaten på att båda
databasernas query optimizers varierade metod för att hämta data beroende på
värdet i predikatet som användes i queryn.