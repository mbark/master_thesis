This chapter contains the results of using the tool to evaluate the two
database's query optimizers. The chapter starts with a section explaining the
results found when evaluating the effect of the cardinality estimate on the access
methods chosen. Following this is a section containing the results for the
second evaluation, which focused on identifying factors other than the
cardinality estimate which cause varying access methods to be selected for the
same relation.

\section{The effect of cardinality estimate}\label{sec:correlation}
This section contains the results of the evaluation which focused on the effect of
cardinality estimates. The section is split into two subsections, describing the
results for the first and second query respectively. The options used for the
evaluation are described in Section~\ref{sec:evaluation} and more specifically
Table~\ref{table:evaluation1}.

The results are shown in form of graphs plotting the number of relations with
varying access methods against the statistics target used when estimating
cardinality. The graphs show how consistent the query optimizer is, the lower
the number of varying access methods --- the less relations are accessed using
different methods. Furthermore, it can be seen from the graphs if there is a
correlation between varying access methods and the the statistics target used.

In addition to the graphs, the full output of the tool is sometimes referred to.
However, due to the lengthiness of this, it can be seen in
Appendix~\ref{appendix:output} instead. The output of the tool can primarily be
used to see what access methods are used --- is it accessed using a full table
scan or via an index? The output is shown in \texttt{JSON} format and is a
stripped down version of the query plans generated by the query optimizer by
\texttt{EXPLAIN}.\@

\subsection{Query \#1}
The first test was done on query \#1, which involves a total of 9 relations,
and the statistics targets were $1$, $d$ and $2d$. The estimation of cardinality
and generation of query plans was repeated a total of 50 times in order to
attempt to capture all access methods that might reasonably be selected.

The results from this test can be seen in Figure~\ref{fig:plot:eval1:query1}.
The graph shows that PostgreSQL has a total of three relations that are accessed
with different methods, but as the statistics target increases that value
decreases to one. For MariaDB the graph shows that it remains consistent in
always having a varying access method for one relation.

The output of the tool shows that for PostgreSQL the relations \textit{cmm},
\textit{cmt} and \textit{est} are the ones that have varying access methods. The
access methods used are either a full table scan --- a \textit{Seq Scan}
--- an index.

For MariaDB the output shows that one relation remains consistent in being
accessed in multiple different ways --- \textit{ct}. Furthermore, the relation
is accessed using three different indexes. As a matter of fact, no relation is
accessed with a full table scan in MariaDB, instead it always selects an index
if one exists.

Finally, the output for MariaDB contains the field \texttt{possible\_keys}, which
shows what keys MariaDB considers possible use to access the data. This shows
that the only relations it considers possible to have different access methods
for are the relations \textit{ct} and \textit{mt}. For all other relations it
only considers one possible access method.

\begin{figure}
  \begin{indexplot}
    \addplot coordinates {
      (0,3) (1,1) (2,1)
    };
    \addplot coordinates {
      (0,1) (1,1) (2,1)
    };
  \end{indexplot}
  \caption[The results when evaluating query \#1 with 50 repetitions and a
  varying statistics target.]{The results when evaluating query \#1 with 50
    repetitions and statistics targets of $1$, $d$ and $2d$, where $d$ is the
    default statistics target for the database. There is a total of 9 relations
    involved in the query.}\label{fig:plot:eval1:query1}
\end{figure}

\subsection{Query \#2}
The second test was done on query \#2, which is a subset of the original query
involving only five relations. The tests were once again done with 50
repetitions and statistics targets of $1$, $d$ and $2d$.

The results of the tests can be seen in Figure~\ref{fig:plot:eval1:query2}. The
graph shows the same results as previously observed for PostgreSQL:\@ a bad
quality cardinality estimate will cause it to vary between access methods.
It is this time the relation \textit{mt} which has varying access methods, once
again varying between a full table scan or using an index. As the statistics
target increases PostgreSQL stabilizes and becomes consistent in always opting
to use an index.

For MariaDB the behavior observed in Figure~\ref{fig:plot:eval1:query1} is once
again observed: the relation \textit{ct} is accessed with three different
indexes and all others using only access method.

\begin{figure}
  \begin{indexplot}
    \addplot coordinates {
      (0,1) (1,0) (2,0)
    };
    \addplot coordinates {
      (0,1) (1,1) (2,1)
    };
  \end{indexplot}
  \caption[The results when evaluating query \#2 with 50 repetitions and a
  varying statistics target.]{The results when evaluating query \#2 with 50
    repetitions and statistics targets of $1$, $d$ and $2d$, where $d$ is the
    default statistics target for the database. There is a total of 5 relations
    involved in the query.}\label{fig:plot:eval1:query2}
\end{figure}

\section{Evaluating subsets of the query}\label{sec:subsets}
This section contains the results for the second evaluation conducted. The focus
of this evaluation was to identify what other factors might affect the choice of
access method if it was not the cardinality estimate. Thus, the tests are done
with only 1 repetition to see if the access methods are different even if the
cardinality estimate is the same for all query plans generated. The options used
for the tests can be seen in Table~\ref{table:evaluation2}.

Three queries are tested: the original query; a subset of the original query
with less tables involved; and a trivial query accessing only the relation
\textit{ct}.

The results are presented in the form of bar charts, with each bin of bars
representing a relation, and each bar representing the number of different access
methods used for that relation for a specific database. Thus, a number larger than 1
shows that the relation is accessed using varying access methods.

\subsection{Query \#1}
The first test was done on query \#1 in order to see which access methods
varied, event though only one query plan was retrieved. The results can be
seen in Figure~\ref{fig:plot:eval2:test1}.

The results show that even though the cardinality estimate remains fixed for all
query plans generated, relations are still accessed with multiple access
methods. For MariaDB it is only the relation ct whereas it is relations
\textit{cmt}, \textit{cmm} and \textit{est} for PostgreSQL.\@

This indicates that there are factors other than the cardinality estimate which
may cause multiple access methods to be used.

\begin{figure}
\begin{indexgraph}
  \addplot coordinates {(ct,3) (t,1) (mt,1) (mm,1) (book,1) (cmt,1) (cmm,1) (est,1) (resamb,1)};
  \addplot coordinates {(ct,1) (t,1) (mt,1) (mm,1) (book,1) (cmt,2) (cmm,2) (est,2) (resamb,1)};
\end{indexgraph}
\caption[The access methods used for query \#1 with 1 repetition.]{The access
  methods when evaluating query \#1 with only 1 repetition. The graph shows that
even though the estimated cardinality is the same for all retrieved query plans,
MariaDB still use different access methods for \textit{ct} and PostgreSQL for
\textit{cmt}, \textit{cmm} and \textit{est}.}\label{fig:plot:eval2:test1}
\end{figure}

\subsection{Query \#3}
The second query tested is simpler than the original as it involves less tables
and thus less joins. The results of the test can be seen in
Figure~\ref{fig:plot:eval2:test2}, which shows that even though the query is
simpler, PostgreSQL still use multiple access methods for the same relations.

The results show that even when the number of relations is reduced somewhat, and
the scenario thus simplified for the query optimizer, the behavior of using
differing access methods remains. This is an indication that the behavior is not
caused by a too complex query, but rather a deliberate choice by the optimizer.

\begin{figure}
\begin{indexgraph}
  \addplot coordinates {(ct,3) (t,1) (mt,1) (mm,1) (book,1) (cmt, 1) (cmm, 1)};
  \addplot coordinates {(ct,1) (t,1) (mt,1) (mm,1) (book,1) (cmt, 2) (cmm, 2)};
\end{indexgraph}
\caption[The access methods used for query \#2 with 1 repetition.]{The access
  methods for query \#2 with 1 repetition. The graph shows that even though the
  query is simpler than the original query, PostgreSQL still use different access
  methods for the relations \textit{cmt} and \textit{cmm}.}\label{fig:plot:eval2:test2}
\end{figure}

\subsection{Query \#3}
The final query used used to test the databases with was the most simple variant
of query \#1 --- selecting and filtering only the relation \textit{ct}.
The results of the test can be seen in Figure~\ref{fig:plot:eval2:test3}, which
shows that MariaDB still use three access methods. Furthermore, the output of
the tool shows that the access methods are the same as for all previous tests
with MariaDB.\@

This test shows quite clearly that even in the most trivial scenario, MariaDB
will still vary between three different access methods. This is a clear
indication that the behavior is intended and that the optimizer calculates it to
be the best one.

\begin{figure}
\begin{indexgraph}
  \addplot coordinates {(ct,3)};
  \addplot coordinates {(ct,1)};
\end{indexgraph}
\caption[The access methods used for query \#3 with 1 repetition.]{The access
  methods used for query \#3 with 1 repetition. Only one relation is accessed,
  making the query the most simple variant of query \#1 --- yet MariaDB will
  still use three different access methods to access
  \textit{ct}.}\label{fig:plot:eval2:test3}
\end{figure}
