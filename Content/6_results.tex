This chapter contains the results of it using the tool to evaluate the two
databases. The chapter starts with a section showing the results when
evaluating the effect of the cardinality estimate for the access methods chosen
by the databases. Following this is a section containing the results of the
second evaluation, which focus on evaluating what factors other than cardinality
estimate cause the databases to select different access methods.

\section{The effect of cardinality estimate}\label{sec:correlation}
This section contains the results of the evaluation which focused on the effect of
cardinality estimates. The section is split into two subsections, describing the
results for the first and second query used for evaluation. The options used
when testing the queries is described in Section~\ref{sec:evaluation} and more
specifically Table~\ref{table:evaluation1}.

To illustrate the results found graphs are used to show the number of relations
with varying access methods. This is then plotted over the statistics
targets used to show what effect improving the quality of the cardinality
estimate has on the access methods used. Thus, the x-axis of the graphs shows
the number of relations with varying access methods and the y-axis shows the
statistics target used.

The graphs can be seen as a measure of how consistent the query optimizer is for
the different statistics targets --- the lower the x-value the more consistent
it is.

In addition to the graphs, the full output of the tool is sometimes referred to,
however due to the lengthiness of this it is included in
Appendix~\ref{appendix:output}. The output of the tool is the relevant parts
of the query plans generated, showing the access methods used for each relation.

\subsection{Query \#1}
The first test was done on query \#1, shown in Figure~\ref{fig:sql:query1}, and
the sample sizes used were $1$, $d$ and $2d$ --- capturing the worst scenario, a
reasonable one and a good one. The estimation of cardinality and generation of
query plans was repeated a total of 50 times, as described in
Table~\ref{table:evaluation1}, in order to attempt to capture all access methods
that might reasonably be selected.

The results from this test can be seen in Figure~\ref{fig:plot:eval1:query1}.
The graph shows that PostgreSQL has a total of three relations that are accessed
with different methods, but as the statistics target increases that value
decreases to one. For MariaDB the graph shows that it remains consistent in
always having a varying access method for one relation.

The output of the tool shows that for PostgreSQL the relations \textit{cmm},
\textit{cmm} and \textit{est} are the ones that have varying access methods. The
access methods used are either a full table scan --- a \textit{Seq Scan}
--- an index.

For MariaDB the output shows that one relation remains consistent in being
accessed in multiple different ways --- \textit{ct}. Furthermore, the relation
is accessed using three different indexes. As a matter of fact, no relation is
accessed with a full table scan in MariaDB, it instead always appears to use an
index if one exists on the relation.

\begin{figure}
  \begin{indexplot}
    \addplot coordinates {
      (0,3) (1,1) (2,1)
    };
    \addplot coordinates {
      (0,1) (1,1) (2,1)
    };
  \end{indexplot}
  \caption[The results when evaluating query \#1 with 50 repetitions and a
  varying statistics target.]{The results when evaluating query \#1 with 50
    repetitions and statistics targets of $1$, $d$ and $2d$, where $d$ is the
    default statistics target for the database.}\label{fig:plot:eval1:query1}
\end{figure}

\subsection{Query \#2}
The second test was done on query \#2, shown in Figure~\ref{fig:sql:query2},
which is a subset of the original query involving only five relations. The
tests were once again done with 50 repetitions and a statistics target of $1$, $d$
and $2d$.

The results of the tests can be seen in Figure~\ref{fig:plot:eval1:query2}. The
graph shows the same results as previously observed for PostgreSQL:\@ a bad
quality cardinality estimate will cause it to vary between access methods.
It is this time the relation \textit{mt} which has varying access methods, once
again varying between a full table scan or using an index. As the statistics
target increases PostgreSQL stabilizes and becomes consistent in always opting
to use an index.

For MariaDB the behavior observed in Figure~\ref{fig:plot:eval1:query1} is also
once again observed: the relation \textit{ct} is accessed with three different
indexes.

\begin{figure}
  \begin{indexplot}
    \addplot coordinates {
      (0,1) (1,0) (2,0)
    };
    \addplot coordinates {
      (0,1) (1,1) (2,1)
    };
  \end{indexplot}
  \caption[The results when evaluating query \#2 with 50 repetitions and a
  varying statistics target.]{The results when evaluating query \#2 with 50
    repetitions and statistics targets of $1$, $d$ and $2d$, where $d$ is the
    default statistics target for the database.}\label{fig:plot:eval1:query2}
\end{figure}

\section{Evaluating subsets of the query}\label{sec:subsets}
This section contains the results for the second evaluation conducted. The focus
of this evaluation was to identify what other factors might affect the choice of
access method if it was not the cardinality estimate. Thus, the tests are done
with only 1 repetition to see if the access methods are different even if the
cardinality estimate is the same for all query plans generated.

Three queries are tested, the original query, a subset of the original query
with less tables involved and a trivial query accessing only the relation
\textit{ct}.

The results are presented in the form of bar charts, with each bin of bars
representing a relation and each bar representing the number of different access
methods used for that relation for a database. Thus, a number larger than 1
shows that the relation is accessed using varying access methods.

\subsection{Query \#1}
The first test was done on query \#1 in order to see which access methods
varied, event though only one query plan was retrieved. The results can be
seen in Figure~\ref{fig:plot:eval2:test1}.

The results show that even though the cardinality estimate remains fixed for all
query plans generated, relations are still accessed with multiple access
methods. For MariaDB it is only the relation ct whereas it is relations
\textit{cmt}, \textit{cmm} and \textit{est} for PostgreSQL.

This indicates that there are factors other than the cardinality estimate which
may cause multiple access methods to be used.

\begin{figure}
\begin{indexgraph}
  \addplot coordinates {(ct,3) (t,1) (mt,1) (mm,1) (book,1) (cmt,1) (cmm,1) (est,1) (resamb,1)};
  \addplot coordinates {(ct,1) (t,1) (mt,1) (mm,1) (book,1) (cmt,2) (cmm,2) (est,2) (resamb,1)};
\end{indexgraph}
\caption[The access methods used for query \#1 with 1 repetition.]{The access
  methods when evaluating query \#1 with only 1 repetition. The graph shows that
even though the estimated cardinality is the same for all retrieved query plans,
MariaDB still use different access methods for \textit{ct} and PostgreSQL for
\textit{cmt}, \textit{cmm} and \textit{est}.}\label{fig:plot:eval2:test1}
\end{figure}

\subsection{Query \#2}
The second query tested is simpler than the original as it involves less tables
and thus less \texttt{JOIN} operations. The results of the test can be seen in
Figure~\ref{fig:plot:eval2:test2}, which shows that even though the query is
simpler, PostgreSQL still use multiple access methods relations.

This indicates that the behavior is intentional and not the cause of the query
being sufficiently complex to throw PostgreSQL off and thus cause it to use
varying access methods. Instead, it seems to some extent deliberate.

\begin{figure}
\begin{indexgraph}
  \addplot coordinates {(ct,3) (t,1) (mt,1) (mm,1) (book,1)};
  \addplot coordinates {(ct,1) (t,1) (mt,1) (mm,1) (book,1)};
\end{indexgraph}
\caption[The access methods used for query \#2 with 1 repetition.]{The access
  methods for query \#2 with 1 repetition. The graph shows that even though the
  query is simpler than the original query, PostgreSQL still use different access
  methods for the relation \textit{cmt}.}\label{fig:plot:eval2:test2}
\end{figure}

\subsection{Query \#3}
The final query used used to test the databases with was the most simple variant
of query \#1 --- selecting and filtering only the relation \textit{ct}.
The results of the test can be seen in Figure~\ref{fig:plot:eval2:test3}, which
shows that MariaDB still use three access methods. Furthermore, the output of
the tool shows that the access methods are the same as for all previous tests
with MariaDB.\@

\begin{figure}
\begin{indexgraph}
  \addplot coordinates {(ct,3)};
  \addplot coordinates {(ct,1)};
\end{indexgraph}
\caption[The access methods used for query \#3 with 1 repetition.]{The access
  methods used for query \#3 with 1 repetition. Only one relation is accessed,
  making the query the most simple variant of query \#1 --- yet MariaDB will
  still use three different access methods to access
  \textit{ct}.}\label{fig:plot:eval2:test3}
\end{figure}
