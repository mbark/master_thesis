This chapter contains the results of using the tool to evaluate the two
databases. The chapter starts with a section showing the results when
evaluating the effect of the cardinality estimate for the access methods chosen
by the databases. Following this is a section containing the results of the
second evaluation, which focus on evaluating what factors other than cardinality
estimate cause the databases to select different access methods.

\section{The effect of cardinality estimate}\label{sec:correlation}
This section contains the results of the evaluation which focus on the effect of
cardinality estimates. The section is split into two sections, one describing
the tests with the low sample size and the other describing the tests with a
high sample size. For the exact parameters used see
Table~\ref{table:evaluation1}, which describes queries tested, repetitions done
and sample sizes used.

The results are presented in the form of graphs showing the number of unique
access methods used for each relation and database. To complement these graphs
are also excerpts of the most relevant parts of the tool's output for the test.
The excerpts focus on the parts of the output showing which access methods are
used for the relations with many access methods.

\subsection{Low sample size}
The low sample size was set to 1 and the process of generating new cardinality
estimates and retrieving query plans for them repeated 50 times.

The first test was done on query \#1 and the results can be seen in
Figure~\ref{fig:plot:eval1:test1}. It is clear from the figure that both
databases use different access methods for the same relation; for MariaDB the
relation \textit{ct} is accessed using three different methods and PostgreSQL
for the relations \textit{mt}, \textit{cmt}, \textit{cmm} and \textit{est} are
all accessed with two different methods.

The access methods used for the relations with many access methods can be seen
in Figure~\ref{fig:json:eval1:test1:mariadb} for MariaDB and
Figure~\ref{fig:json:eval1:test1:postgresql} for PostgreSQL.\@ From these
it can be seen that MariaDB varies between three different indexes when
accessing \textit{ct}. For PostgreSQL it is instead the case that the it varies
between using a full table scan (called a \textit{seq scan}) or an index.

The results of the second test --- using query \#2 instead --- can be seen in
Figure~\ref{fig:plot:eval1:test2}. The query  is simpler in terms of the number
of joins and tables involved, yet it is still the case that PostgreSQL will
select multiple access methods for the relation \textit{mt}.

\begin{figure}[ht]
\begin{indexgraph}
  \addplot coordinates {(ct,3) (t,1) (mt,1) (mm,1) (book,1) (cmt,1) (cmm,1) (est,1) (resamb,1)};
  \addplot coordinates {(ct,1) (t,1) (mt,2) (mm,1) (book,1) (cmt,2) (cmm,2) (est,2) (resamb,1)};
\end{indexgraph}
\caption[The access methods used for query \#1 with 50 repetitions and a sample
size of 1.]{The test conducted in the first evaluation using a sample size of 1
  and 50 repetitions. The graph shows that MariaDB varies between three
  different access methods for the relation \textit{ct} and PostgreSQL between
  two for the relations \textit{mt}, \textit{cmt}, \textit{cmm} and
  \textit{est}.}\label{fig:plot:eval1:test1}
\end{figure}

\begin{figure}[ht]
  \begin{minted}[breaklines, breakanywhere, fontsize=\footnotesize]{json}
{
    "ct": [
        {
            "table": "ct",
            "key": "match_view_ref",
            "possible_keys": "match_view_ref,match_view_ref_2,CycleTrade_master_trade_ref,CycleTrade_trade_ref,CycleTrade_match_view_ref"
        },
        {
            "table": "ct",
            "key": "match_view_ref_2",
            "possible_keys": "match_view_ref,match_view_ref_2,CycleTrade_master_trade_ref,CycleTrade_trade_ref,CycleTrade_match_view_ref"
        },
        {
            "table": "ct",
            "key": "CycleTrade_match_view_ref",
            "possible_keys": "match_view_ref,match_view_ref_2,CycleTrade_master_trade_ref,CycleTrade_trade_ref,CycleTrade_match_view_ref"
        }
    ]
}
\end{minted}
  \caption[Excerpt of the output for MariaDB, query \#1, 50 repetitions and a
  sample size of 1.]{Excerpt of the tool's output when evaluating MariaDB with
    50 repetitions and a sample size of 1. The relation shown, \textit{ct}, is
    the one which MariaDB use different access methods for.}\label{fig:json:eval1:test1:mariadb}
\end{figure}

\begin{figure}[ht]
  \begin{minted}[breaklines, breakanywhere, fontsize=\footnotesize]{json}
{
    "cmm": [
        {
            "Node Type": "Index Scan",
            "Index Name": "\"NoneCycleMasterMatch_cycle_master_match_key_pkey\"",
            "Alias": "cmm"
        },
        {
            "Node Type": "Seq Scan",
            "Alias": "cmm"
        }
    ],
    "cmt": [
        {
            "Node Type": "Index Scan",
            "Index Name": "\"NoneCycleMasterTrade_cycle_master_trade_key_pkey\"",
            "Alias": "cmt"
        },
        {
            "Node Type": "Seq Scan",
            "Alias": "cmt"
        }
    ],
    "est": [
        {
            "Node Type": "Index Scan",
            "Index Name": "\"NoneExternalServiceTrade_cycle_master_trade_ref_external_servic\"",
            "Alias": "est"
        },
        {
            "Node Type": "Seq Scan",
            "Alias": "est"
        }
    ],
    "mt": [
        {
            "Node Type": "Seq Scan",
            "Alias": "mt"
        },
        {
            "Node Type": "Index Scan",
            "Index Name": "\"NoneMasterTrade_master_trade_key_pkey\"",
            "Alias": "mt"
        }
    ]
}\end{minted}
  \caption[Excerpt of the tool's output for PostgreSQL, query \#1, 50 repetitions and a
  sample size of 1.]{Excerpt of the tool's output when evaluating PostgreSQL
    with 50 repetitions and a sample size of 1. The relation shown --- cmm, cmt,
    est and mt --- are those which PostgreSQL use different access methods
    for. The output shows that compared to MariaDB, PostgreSQL only varies
    between using an index or doing a full table scan --- never between several
    different indexes.}\label{fig:json:eval1:test1:postgresql}
\end{figure}

\begin{figure}[ht]
\begin{indexgraph}
  \addplot coordinates {(ct,3) (t,1) (mt,1) (mm,1) (book,1)};
  \addplot coordinates {(ct,1) (t,1) (mt,2) (mm,1) (book,1)};
\end{indexgraph}
\caption[The access methods used for the query \#2 with 50 repetitions and a sample
size of 1.]{The test conducted using query \#2, 50 repetitions
  and a sample size of 1 to simulate a cardinality estimate of low
  quality. The graph shows that PostgreSQL will vary between using two different
  access methods for the relation \textit{mt}, even though the query is simpler
  in terms of the number of joins used compared to query
  \#1.}\label{fig:plot:eval1:test2}
\end{figure}

\subsection{High sample size}
The second set of tests done was done using a sample size of 100, the number of
repetitions was set to the same value as before --- 50.

The first test was done on query \#1 and the results of it can be seen in
Figure~\ref{fig:plot:eval1:test3}. Compared to the access methods used for the
low sample size, seen in Figure~\ref{fig:plot:eval1:test1}, PostgreSQL has
become more consistent in its choice of access methods. However, MariaDB remains
consistent in using three different access methods for the relation ct.

An excerpt of the tool's output for the test on PostgreSQL can be seen in
Figure~\ref{fig:json:eval1:test3:postgresql}. From the output it is clear that
effect of improved cardinality estimated made PostgreSQL never choose to do a
full table scan rather than use an index for all relations shown.

The output for MariaDB can be seen in Figure~\ref{fig:json:eval1:test3:mariadb}
and comparing it to the output from the test with a low sample size, seen in
Figure~\ref{fig:json:eval1:test1:mariadb}, it is clear that the same three
access methods are used --- regardless of sample size used when estimating cardinality.

Query \#2 was also tested with a high sample size and the results, seen in
Figure~\ref{fig:plot:eval1:test4}, show the same results as those for PostgreSQL
and query \#1 --- the relation \textit{mt} is now accessed with only one access
method.

\begin{figure}[ht]
\begin{indexgraph}
  \addplot coordinates {(ct,3) (t,1) (mt,1) (mm,1) (book,1) (cmt,1) (cmm,1) (est,1) (resamb,1)};
  \addplot coordinates {(ct,1) (t,1) (mt,1) (mm,1) (book,1) (cmt,1) (cmm,2) (est,1) (resamb,1)};
\end{indexgraph}
\caption[The access methods used for query \#1 with 50 repetitions and a sample
size of 100.]{The test conducted using query \#1, 50 repetitions
  and a sample size of 100 to simulate a cardinality estimate of high
  quality. The graph once again shows that MariaDB use three different access
  methods for the relation \textit{ct.} PostgreSQL on the other hand now only use
  different access methods for the relation \textit{cmm} and none else.}\label{fig:plot:eval1:test3}
\end{figure}

\begin{figure}[ht]
  \begin{minted}[breaklines, breakanywhere, fontsize=\footnotesize]{json}
{
    "ct": [
        {
            "table": "ct",
            "key": "match_view_ref",
            "possible_keys": "match_view_ref,match_view_ref_2,CycleTrade_master_trade_ref,CycleTrade_trade_ref,CycleTrade_match_view_ref"
        },
        {
            "table": "ct",
            "key": "match_view_ref_2",
            "possible_keys": "match_view_ref,match_view_ref_2,CycleTrade_master_trade_ref,CycleTrade_trade_ref,CycleTrade_match_view_ref"
        },
        {
            "table": "ct",
            "key": "CycleTrade_match_view_ref",
            "possible_keys": "match_view_ref,match_view_ref_2,CycleTrade_master_trade_ref,CycleTrade_trade_ref,CycleTrade_match_view_ref"
        }
    ]
  }
\end{minted}
  \caption[Excerpt of the tool's output for MariaDB, query \#1, 50 repetitions and a
  sample size of 100.]{Excerpt of the tool's output when evaluating MariaDB
    with query \#1, 50 repetitions and a sample size of 100. Comparing this
    output to that given when using a sample size of 1, shown in
    Figure~\ref{fig:json:eval1:test1:mariadb}, shows that the same access
    methods are used --- regardless of sample size used when estimating
    cardinality.}\label{fig:json:eval1:test3:mariadb}
\end{figure}

\begin{figure}[ht]
  \begin{minted}[breaklines, breakanywhere, fontsize=\footnotesize]{json}
{
    "cmm": [
        {
            "Node Type": "Index Scan",
            "Index Name": "\"NoneCycleMasterMatch_cycle_master_match_key_pkey\"",
            "Alias": "cmm"
        },
        {
            "Node Type": "Seq Scan",
            "Alias": "cmm"
        }
    ],
    "cmt": [
        {
            "Node Type": "Index Scan",
            "Index Name": "\"NoneCycleMasterTrade_cycle_master_trade_key_pkey\"",
            "Alias": "cmt"
        }
    ],
    "est": [
        {
            "Node Type": "Index Scan",
            "Index Name": "\"NoneExternalServiceTrade_cycle_master_trade_ref_external_servic\"",
            "Alias": "est"
        }
    ],
    "mt": [
        {
            "Node Type": "Index Scan",
            "Index Name": "\"NoneMasterTrade_master_trade_key_pkey\"",
            "Alias": "mt"
        }
    ]
}
\end{minted}
  \caption[Excerpt of the tool's output for PostgreSQL, query \#1, 50 repetitions and a
  sample size of 100.]{Excerpt of the tool's output when evaluating PostgreSQL
    with query \#1, 50 repetitions and a sample size of 100. The relations shown are the
    same as those for the test with a sample size of 1. The output shows that
    all but the relation \textit{cmm} are now accessed by using an
    index.}\label{fig:json:eval1:test3:postgresql}
\end{figure}

\begin{figure}[ht]
\begin{indexgraph}
  \addplot coordinates {(ct,3) (t,1) (mt,1) (mm,1) (book,1)};
  \addplot coordinates {(ct,1) (t,1) (mt,1) (mm,1) (book,1)};
\end{indexgraph}
\caption[The access methods used for the query \#2 with 50 repetitions and a sample
size of 100.]{The test conducted using query \#2, 50 repetitions
  and a sample size of 100 to simulate a cardinality estimate of high
  quality. The graph should be compared that of the test with a sample size of
  1, which can be seen in Figure~\ref{fig:plot:eval1:test2}. The notable
  difference is that \textit{mt} is selected using only one access method,
  rather than two.}\label{fig:plot:eval1:test4}
\end{figure}

\section{Evaluating subsets of the query}\label{sec:subsets}
This section contains the results for the second evaluation conducted. The focus
of this evaluation was to identify what other factors might affect the choice of
access method if it was not the cardinality estimate. Thus, the tests are done
with only 1 repetition to see if the access methods are different even if the
cardinality estimate is the same for all query plans generated.

Three queries are tested, the original query, a subset of the original query
with less tables involved and a trivial query accessing only the relation
\textit{ct}.

\subsection{Query \#1}
The first test was done on query \#1 in order to see which access methods
varied, event though only one query plan was retrieved and the results can be
seen in Figure~\ref{fig:plot:eval2:test1}. The results show that even though the
cardinality estimate is fixed, different query plans are still generated.

The tool's output shows that MariaDB accesses the relation \textit{ct} with
three different access methods (as previously observed in
Figure~\ref{fig:json:eval1:test1:mariadb} and
Figure~\ref{fig:json:eval1:test3:mariadb}).

The output for PostgreSQL can be seen in
Figure~\ref{fig:json:eval2:test1:postgresql} and shows that relations
\textit{cmt}, \textit{cmm} and \textit{est} are accessed with different access
methods. Unlike in the previous evaluation the relation \textit{mt} is not
accessed differently.

\begin{figure}[ht]
\begin{indexgraph}
  \addplot coordinates {(ct,3) (t,1) (mt,1) (mm,1) (book,1) (cmt,1) (cmm,1) (est,1) (resamb,1)};
  \addplot coordinates {(ct,1) (t,1) (mt,1) (mm,1) (book,1) (cmt,2) (cmm,2) (est,2) (resamb,1)};
\end{indexgraph}
\caption[The access methods used for query \#1 with 1 repetition.]{The access
  methods when evaluating query \#1 with only 1 repetition. The graph shows that
even though the estimated cardinality is the same for all retrieved query plans,
MariaDB still use different access methods for \textit{ct} and PostgreSQL for
\textit{cmt}, \textit{cmm} and \textit{est}.}\label{fig:plot:eval2:test1}
\end{figure}

\begin{figure}[ht]
  \begin{minted}[breaklines, breakanywhere, fontsize=\footnotesize]{json}
{
    "ct": [
        {
            "table": "ct",
            "key": "match_view_ref",
            "possible_keys": "match_view_ref,match_view_ref_2,CycleTrade_master_trade_ref,CycleTrade_trade_ref,CycleTrade_match_view_ref"
        },
        {
            "table": "ct",
            "key": "match_view_ref_2",
            "possible_keys": "match_view_ref,match_view_ref_2,CycleTrade_master_trade_ref,CycleTrade_trade_ref,CycleTrade_match_view_ref"
        },
        {
            "table": "ct",
            "key": "CycleTrade_match_view_ref",
            "possible_keys": "match_view_ref,match_view_ref_2,CycleTrade_master_trade_ref,CycleTrade_trade_ref,CycleTrade_match_view_ref"
        }
    ]
}
\end{minted}
\caption[Excerpt of the tool's output when testing MariaDB with query \#1 and 1
repetition.]{Excerpt of the tool's output when testing MariaDB with query \#1
  and 1 repetition. Only the relation which MariaDB select multiple different
  access methods for, \textit{ct}, is
  shown.}\label{fig:json:eval2:test1:mariadb}
\end{figure}

\begin{figure}[ht]
  \begin{minted}[breaklines, breakanywhere, fontsize=\footnotesize]{json}
    {
    "cmt": [
        {
            "Node Type": "Index Scan",
            "Index Name": "\"NoneCycleMasterTrade_cycle_master_trade_key_pkey\"",
            "Alias": "cmt"
        },
        {
            "Node Type": "Seq Scan",
            "Alias": "cmt"
        }
    ]
}
\end{minted}
\caption[Excerpt of the tool's output when testing PostgreSQL with query \#1 and
1 repetition.]{Excerpt of the tool's output when testing PostgreSQL with query
  \#1 and 1 repetition. The relation \textit{cmt} is shown as PostgreSQL selects
  multiple access methods for it and the relation is part of query \#2, which is
  used for the second test.}\label{fig:json:eval2:test1:postgresql}
\end{figure}

\subsection{Query \#2}
The second query tested is simpler than the original as it involves less tables
and thus \texttt{JOIN} operations. The results of the test can be seen in
Figure~\ref{fig:plot:eval2:test2}, which shows that PostgreSQL still use
multiple access methods for the relation \textit{cmt}.

The output of the tool for PostgreSQL can be seen in
Figure~\ref{fig:json:eval2:test2:postgresql}. Comparing the output to that of
the test on the original query, shown in
Figure~\ref{fig:json:eval2:test1:postgresql}, shows that \textit{cmt} is
accessed with the same two access methods for both.

\begin{figure}[ht]
\begin{indexgraph}
  \addplot coordinates {(ct,3) (t,1) (mt,1) (mm,1) (book,1) (cmt,1) };
  \addplot coordinates {(ct,1) (t,1) (mt,1) (mm,1) (book,1) (cmt,2) };
\end{indexgraph}
\caption[The access methods used for query \#2 with 1 repetition.]{The access
  methods for query \#2 with 1 repetition. The graph shows that even though the
  query is simpler than the original query, PostgreSQL still use different access
  methods for the relation \textit{cmt}.}\label{fig:plot:eval2:test2}
\end{figure}

\begin{figure}[ht]
  \begin{minted}[breaklines, breakanywhere, fontsize=\footnotesize]{json}
    {
      "cmt": [
      {
        "Node Type": "Index Scan",
        "Index Name": "\"NoneCycleMasterTrade_cycle_master_trade_key_pkey\"",
        "Alias": "cmt"
      },
      {
        "Node Type": "Seq Scan",
        "Alias": "cmt"
      }
      ]
    }
\end{minted}
  \caption[Excerpt of the tool's output when testing PostgreSQL with query \#2 and 1
  repetition.]{Excerpt of the tool's output when evaluating PostgreSQL with
    query \#2 and 1 repetition. The relation shown is the same as in
    Figure~\ref{fig:json:eval2:test1:postgresql} and it can once again be seen
    that the same two access methods are used by PostgreSQL.}\label{fig:json:eval2:test2:postgresql}
\end{figure}

\subsection{Query \#3}
The final query used used to test the databases with was the most simple variant
of query \#1 --- selecting and filtering only the relation \textit{ct}.
The results of the test can be seen in Figure~\ref{fig:plot:eval2:test3}, which
shows that MariaDB still use three access methods. Furthermore, the output of
the tool, shown in Figure~\ref{fig:json:eval2:test3:mariadb}, shows that it is
the same methods as those used for the full query, shown in
Figure~\ref{fig:json:eval2:test1:mariadb}.

\begin{figure}[ht]
\begin{indexgraph}
  \addplot coordinates {(ct,3)};
  \addplot coordinates {(ct,1)};
\end{indexgraph}
\caption[The access methods used for query \#3 with 1 repetition.]{The access
  methods used for query \#3 with 1 repetition. Only one relation is accessed,
  making the query the most simple variant of query \#1 --- yet MariaDB will
  still use three different access methods to access
  \textit{ct}.}\label{fig:plot:eval2:test3}
\end{figure}

\begin{figure}[ht]
  \begin{minted}[breaklines, breakanywhere, fontsize=\footnotesize]{json}
{
    "ct": [
        {
            "table": "ct",
            "key": "match_view_ref",
            "possible_keys": "match_view_ref,match_view_ref_2,CycleTrade_match_view_ref"
        },
        {
            "table": "ct",
            "key": "match_view_ref_2",
            "possible_keys": "match_view_ref,match_view_ref_2,CycleTrade_match_view_ref"
        },
        {
            "table": "ct",
            "key": "CycleTrade_match_view_ref",
            "possible_keys": "match_view_ref,match_view_ref_2,CycleTrade_match_view_ref"
        }
    ]
}
\end{minted}
  \caption[Excerpt of the tool's output when testing MariaDB with query \#3 and 1
  repetition.]{Excerpt of the tool's output when evaluating MariaDB with query
    \#3 and 1 repetition. The output shows that the access methods for the
    relation \textit{ct} are the same for query \#3 and query
    \#1.}\label{fig:json:eval2:test3:mariadb}
\end{figure}
