This chapter contains the results of using the tool to evaluate the two
databases. The chapter starts with a section showing the results when
evaluating the effect of the cardinality estimate for the access methods chosen
by the databases. Following this is a section containing the results of the
second evaluation, which focus on evaluating what factors other than cardinality
estimate cause the databases to select different access methods.

\section{The effect of cardinality estimate}\label{sec:correlation}
This section contains the results of the evaluation which focused on the effect of
cardinality estimates. The section is split into two sections, one describing
the tests with the low sample size and the other describing the tests with a
high sample size. For the exact parameters used see
Table~\ref{table:evaluation1}, which describes queries tested, repetitions done
and sample sizes used.

The results are presented in the form of graphs showing the number of unique
access methods used for each relation and database. To complement these graphs
are also excerpts of the most relevant parts of the tool's output for the test.
The excerpts focus on the parts of the output showing which access methods are
used for the relations with many access methods.

\subsection{Low sample size}
The low sample size was set to 1 and the process of generating new cardinality
estimates and retrieving query plans for them repeated 50 times.

The first test was done on query \#1 and the results can be seen in
Figure~\ref{fig:plot:eval1:test1}. It is clear from the figure that both
databases use different access methods for the same relation; for MariaDB the
relation \textit{ct} is accessed using three different methods and PostgreSQL
for the relations \textit{mt}, \textit{cmt}, \textit{cmm} and \textit{est} are
all accessed with two different methods.

The output of the tool, shown in Appendix~\ref{appendix:output}, shows that
MariaDB access the relation \textit{ct} with three different indexes. The
output also shows that PostgreSQL on the other hand only varies between doing a
full table scan or using an index for all the relations it has varying access
methods for.

The results of the second test --- using query \#2 instead --- can be seen in
Figure~\ref{fig:plot:eval1:test2}. The query  is simpler in terms of the number
of joins and tables involved, yet it is still the case that PostgreSQL will
select multiple access methods for the relation \textit{mt}.

\begin{figure}
\begin{indexgraph}
  \addplot coordinates {(ct,3) (t,1) (mt,1) (mm,1) (book,1) (cmt,1) (cmm,1) (est,1) (resamb,1)};
  \addplot coordinates {(ct,1) (t,1) (mt,2) (mm,1) (book,1) (cmt,2) (cmm,2) (est,2) (resamb,1)};
\end{indexgraph}
\caption[The access methods used for query \#1 with 50 repetitions and a sample
size of 1.]{The test conducted in the first evaluation using a sample size of 1
  and 50 repetitions. The graph shows that MariaDB varies between three
  different access methods for the relation \textit{ct} and PostgreSQL between
  two for the relations \textit{mt}, \textit{cmt}, \textit{cmm} and
  \textit{est}.}\label{fig:plot:eval1:test1}
\end{figure}

\begin{figure}
\begin{indexgraph}
  \addplot coordinates {(ct,3) (t,1) (mt,1) (mm,1) (book,1)};
  \addplot coordinates {(ct,1) (t,1) (mt,2) (mm,1) (book,1)};
\end{indexgraph}
\caption[The access methods used for the query \#2 with 50 repetitions and a sample
size of 1.]{The test conducted using query \#2, 50 repetitions
  and a sample size of 1 to simulate a cardinality estimate of low
  quality. The graph shows that PostgreSQL will vary between using two different
  access methods for the relation \textit{mt}, even though the query is simpler
  in terms of the number of joins used compared to query
  \#1.}\label{fig:plot:eval1:test2}
\end{figure}

\subsection{High sample size}
The second set of tests done was done using a sample size of 100, the number of
repetitions was set to the same value as before --- 50.

The first test was done on query \#1 and the results of it can be seen in
Figure~\ref{fig:plot:eval1:test3}. Compared to the access methods used for the
low sample size, seen in Figure~\ref{fig:plot:eval1:test1}, PostgreSQL has
become more consistent in its choice of access methods. However, MariaDB remains
consistent in using three different access methods for the relation ct.

Furthermore, the output of the tool (shown in Appendix~\ref{appendix:output})
shows that as the cardinality estimate is improved, PostgreSQL will stop doing
full table scans and instead always use an index. The output also shows that the
access methods used for MariaDB remains the same three indexes.

Query \#2 was also tested with a high sample size and the results, seen in
Figure~\ref{fig:plot:eval1:test4}, show the same results as those for PostgreSQL
and query \#1 --- the relation \textit{mt} is now accessed with only one access
method.

\begin{figure}
\begin{indexgraph}
  \addplot coordinates {(ct,3) (t,1) (mt,1) (mm,1) (book,1) (cmt,1) (cmm,1) (est,1) (resamb,1)};
  \addplot coordinates {(ct,1) (t,1) (mt,1) (mm,1) (book,1) (cmt,1) (cmm,2) (est,1) (resamb,1)};
\end{indexgraph}
\caption[The access methods used for query \#1 with 50 repetitions and a sample
size of 100.]{The test conducted using query \#1, 50 repetitions
  and a sample size of 100 to simulate a cardinality estimate of high
  quality. The graph once again shows that MariaDB use three different access
  methods for the relation \textit{ct.} PostgreSQL on the other hand now only use
  different access methods for the relation \textit{cmm} and none else.}\label{fig:plot:eval1:test3}
\end{figure}

\begin{figure}
\begin{indexgraph}
  \addplot coordinates {(ct,3) (t,1) (mt,1) (mm,1) (book,1)};
  \addplot coordinates {(ct,1) (t,1) (mt,1) (mm,1) (book,1)};
\end{indexgraph}
\caption[The access methods used for the query \#2 with 50 repetitions and a sample
size of 100.]{The test conducted using query \#2, 50 repetitions
  and a sample size of 100 to simulate a cardinality estimate of high
  quality. The graph should be compared that of the test with a sample size of
  1, which can be seen in Figure~\ref{fig:plot:eval1:test2}. The notable
  difference is that \textit{mt} is selected using only one access method,
  rather than two.}\label{fig:plot:eval1:test4}
\end{figure}

\section{Evaluating subsets of the query}\label{sec:subsets}
This section contains the results for the second evaluation conducted. The focus
of this evaluation was to identify what other factors might affect the choice of
access method if it was not the cardinality estimate. Thus, the tests are done
with only 1 repetition to see if the access methods are different even if the
cardinality estimate is the same for all query plans generated.

Three queries are tested, the original query, a subset of the original query
with less tables involved and a trivial query accessing only the relation
\textit{ct}.

\subsection{Query \#1}
The first test was done on query \#1 in order to see which access methods
varied, event though only one query plan was retrieved and the results can be
seen in Figure~\ref{fig:plot:eval2:test1}. The results show that even though the
cardinality estimate is fixed, different query plans are still generated.

Furthermore, the graph shows that PostgreSQL will access the relations
\textit{cmt}, \textit{cmm} and \textit{est} with varying access methods, however
--- unlikely in the previous evaluation (shown in
Figure~\ref{fig:plot:eval1:test1}), the relation \textit{mt} is not accessed
with different access methods.

\begin{figure}
\begin{indexgraph}
  \addplot coordinates {(ct,3) (t,1) (mt,1) (mm,1) (book,1) (cmt,1) (cmm,1) (est,1) (resamb,1)};
  \addplot coordinates {(ct,1) (t,1) (mt,1) (mm,1) (book,1) (cmt,2) (cmm,2) (est,2) (resamb,1)};
\end{indexgraph}
\caption[The access methods used for query \#1 with 1 repetition.]{The access
  methods when evaluating query \#1 with only 1 repetition. The graph shows that
even though the estimated cardinality is the same for all retrieved query plans,
MariaDB still use different access methods for \textit{ct} and PostgreSQL for
\textit{cmt}, \textit{cmm} and \textit{est}.}\label{fig:plot:eval2:test1}
\end{figure}

\subsection{Query \#2}
The second query tested is simpler than the original as it involves less tables
and thus \texttt{JOIN} operations. The results of the test can be seen in
Figure~\ref{fig:plot:eval2:test2}, which shows that PostgreSQL still use
multiple access methods for the relation \textit{cmt}.

\begin{figure}
\begin{indexgraph}
  \addplot coordinates {(ct,3) (t,1) (mt,1) (mm,1) (book,1) (cmt,1) };
  \addplot coordinates {(ct,1) (t,1) (mt,1) (mm,1) (book,1) (cmt,2) };
\end{indexgraph}
\caption[The access methods used for query \#2 with 1 repetition.]{The access
  methods for query \#2 with 1 repetition. The graph shows that even though the
  query is simpler than the original query, PostgreSQL still use different access
  methods for the relation \textit{cmt}.}\label{fig:plot:eval2:test2}
\end{figure}

\subsection{Query \#3}
The final query used used to test the databases with was the most simple variant
of query \#1 --- selecting and filtering only the relation \textit{ct}.
The results of the test can be seen in Figure~\ref{fig:plot:eval2:test3}, which
shows that MariaDB still use three access methods. Furthermore, the output of
the tool (shown in Appendix~\ref{appendix:output}) shows that the access methods
are the same as for all previous tests with MariaDB.

\begin{figure}
\begin{indexgraph}
  \addplot coordinates {(ct,3)};
  \addplot coordinates {(ct,1)};
\end{indexgraph}
\caption[The access methods used for query \#3 with 1 repetition.]{The access
  methods used for query \#3 with 1 repetition. Only one relation is accessed,
  making the query the most simple variant of query \#1 --- yet MariaDB will
  still use three different access methods to access
  \textit{ct}.}\label{fig:plot:eval2:test3}
\end{figure}
