This chapter contains the results of using the tool to evaluate the two
databases. This chapter will start with a section showing the results when
evaluating the databases to identify a correlation between sample size and
selection of different access methods. Following this are further results
showing the results when evaluating subsets of the original query.

\section{Correlation between sample size and selection of access methods}\label{sec:correlation}
This section contains the results of the evaluation done in order to determine a
correlation between sample size and the number of different access methods used
to access the same relation. The evaluation consisted of a total of four tests,
two tests per database. For the options used in the tests see
Section~\ref{sec:evaluation} and more specifically
Table~\ref{table:evaluation1}. The results will be presented in the form of two
bar charts, one for each of the sample sizes used. The full output of the
program, showing which access methods are actually used, can be seen in Appendix~\ref{appendix:output}.

The first test done can be seen in Figure~\ref{fig:plot:eval1:test1}. Each
relation involved in the query is represented by a three different bars --- one
for the number of possible access methods and one for the actual access methods
used for either of the two databases. From the figure it can be seen that there
are six possible access methods for \texttt{ct} (one full table scan and five
usable indexes) and that MariaDB will vary between three of these methods,
whereas PostgreSQL consistently selects the same one.

The second evaluation of query \#1, using a high sample size, can be seen in
Figure~\ref{fig:plot:eval1:test2}, which can be read in the same way as the
previous figure. The graph has three bars per relation once again.

\begin{figure}
\begin{indexgraph}
  \addplot coordinates {(ct,6) (t,2) (mt,3) (mm,2) (book,2) (cmt,2) (cmm,2) (est,2) (resamb,2)};
  \addplot coordinates {(ct,3) (t,1) (mt,1) (mm,1) (book,1) (cmt,1) (cmm,1) (est,1) (resamb,1)};
  \addplot coordinates {(ct,1) (t,1) (mt,2) (mm,1) (book,1) (cmt,2) (cmm,2) (est,2) (resamb,1)};
\end{indexgraph}
\caption[The access methods used with a low sample size.]{The test conducted in
  the first evaluation using a small sample size and many repetitions. See
  Table~\ref{table:evaluation1} for the exact values used.}\label{fig:plot:eval1:test1}
\end{figure}

\begin{figure}
\begin{indexgraph}
  \addplot coordinates {(ct,6) (t,2) (mt,3) (mm,2) (book,2) (cmt,2) (cmm,2) (est,2) (resamb,2)};
  \addplot coordinates {(ct,3) (t,1) (mt,1) (mm,1) (book,1) (cmt,1) (cmm,1) (est,1) (resamb,1)};
  \addplot coordinates {(ct,1) (t,1) (mt,1) (mm,1) (book,1) (cmt,1) (cmm,2) (est,1) (resamb,1)};
\end{indexgraph}
\caption[The access methods used with a high sample size.]{The test conducted in
  the first evaluation using a large sample size and many repetitions. See
  Table~\ref{table:evaluation1} for the exact values used.}\label{fig:plot:eval1:test2}
\end{figure}

\section{Evaluating subsets of the query}\label{sec:subsets}
This section contains the results for the second evaluation conducted. The evaluation
was done on queries \#1 through \#9, where \#2 through \#9 are subsets of query
\#1 and the parameters used for these tests can be seen in Figure~\ref{table:evaluation2}.

The results of the evaluation can be seen in
Figures~\ref{fig:plot:eval2:test1}~---~\ref{fig:plot:eval2:test9}. The graphs
are formatted the same as those described in Section~\ref{sec:correlation}.
Queries \#2 --- \#9 are subsets of the original query and contain fewer tables,
which is reflected in the fact that the graphs contain successively fewer bars.

\begin{figure}[ht]
\begin{indexgraph}
  \addplot coordinates {(ct,6) (t,2) (mt,3) (mm,2) (book,2) (cmt,2) (cmm,2) (est,2) (resamb,2)};
  \addplot coordinates {(ct,3) (t,1) (mt,1) (mm,1) (book,1) (cmt,1) (cmm,1) (est,1) (resamb,1)};
  \addplot coordinates {(ct,1) (t,1) (mt,1) (mm,1) (book,1) (cmt,2) (cmm,2) (est,2) (resamb,1)};
\end{indexgraph}
\caption[The index selections for query \#1.]{The index selections for query \#1
are shown next to each other, showing the actual index selections next to the
possible index selections.}\label{fig:plot:eval2:test1}
\end{figure}

\begin{figure}
\begin{indexgraph}
  \addplot coordinates {(ct,6) (t,2) (mt,3) (mm,2) (book,2) (cmt,2) (cmm,2) (est,2)};
  \addplot coordinates {(ct,3) (t,1) (mt,1) (mm,1) (book,1) (cmt,1) (cmm,1) (est,1)};
  \addplot coordinates {(ct,1) (t,1) (mt,1) (mm,1) (book,1) (cmt,2) (cmm,2) (est,2)};
\end{indexgraph}
\caption[The index selections for query \#2.]{The index selections for query \#2
are shown next to each other, showing the actual index selections next to the
possible index selections.}\label{fig:plot:eval2:test2}
\end{figure}

\begin{figure}
\begin{indexgraph}
  \addplot coordinates {(ct,6) (t,2) (mt,3) (mm,2) (book,2) (cmt,2) (cmm,2) };
  \addplot coordinates {(ct,3) (t,1) (mt,1) (mm,1) (book,1) (cmt,1) (cmm,1) };
  \addplot coordinates {(ct,1) (t,1) (mt,1) (mm,1) (book,1) (cmt,2) (cmm,2) };
\end{indexgraph}
\caption[The index selections for query \#3.]{The index selections for query \#3
are shown next to each other, showing the actual index selections next to the
possible index selections.}\label{fig:plot:eval2:test3}
\end{figure}

\begin{figure}
\begin{indexgraph}
  \addplot coordinates {(ct,6) (t,2) (mt,3) (mm,2) (book,2) (cmt,2) };
  \addplot coordinates {(ct,3) (t,1) (mt,1) (mm,1) (book,1) (cmt,1) };
  \addplot coordinates {(ct,1) (t,1) (mt,1) (mm,1) (book,1) (cmt,2) };
\end{indexgraph}
\caption[The index selections for query \#4.]{The index selections for query \#4
are shown next to each other, showing the actual index selections next to the
possible index selections.}\label{fig:plot:eval2:test4}
\end{figure}

\begin{figure}
\begin{indexgraph}
  \addplot coordinates {(ct,6) (t,2) (mt,3) (mm,2) (book,2) };
  \addplot coordinates {(ct,3) (t,1) (mt,1) (mm,1) (book,1) };
  \addplot coordinates {(ct,1) (t,1) (mt,1) (mm,1) (book,1) };
\end{indexgraph}
\caption[The index selections for query \#5.]{The index selections for query \#5
are shown next to each other, showing the actual index selections next to the
possible index selections.}\label{fig:plot:eval2:test5}
\end{figure}

\begin{figure}
\begin{indexgraph}
  \addplot coordinates {(ct,6) (t,2) (mt,3) (mm,2) };
  \addplot coordinates {(ct,3) (t,1) (mt,1) (mm,1) };
  \addplot coordinates {(ct,1) (t,1) (mt,1) (mm,1) };
\end{indexgraph}
\caption[The index selections for query \#6.]{The index selections for query \#6
are shown next to each other, showing the actual index selections next to the
possible index selections.}\label{fig:plot:eval2:test6}
\end{figure}

\begin{figure}
\begin{indexgraph}
  \addplot coordinates {(ct,6) (t,2) (mt,3) };
  \addplot coordinates {(ct,3) (t,1) (mt,1) };
  \addplot coordinates {(ct,1) (t,1) (mt,1) };
\end{indexgraph}
\caption[The index selections for query \#7.]{The index selections for query \#7
are shown next to each other, showing the actual index selections next to the
possible index selections.}\label{fig:plot:eval2:test7}
\end{figure}

\begin{figure}
\begin{indexgraph}
  \addplot coordinates {(ct,6) (t,2) };
  \addplot coordinates {(ct,3) (t,1) };
  \addplot coordinates {(ct,1) (t,1) };
\end{indexgraph}
\caption[The index selections for query \#8.]{The index selections for query \#8
are shown next to each other, showing the actual index selections next to the
possible index selections.}\label{fig:plot:eval2:test8}
\end{figure}

\begin{figure}
\begin{indexgraph}
  \addplot coordinates {(ct,6)};
  \addplot coordinates {(ct,3)};
  \addplot coordinates {(ct,1)};
\end{indexgraph}
\caption[The index selections for query \#9.]{The index selections for query \#9
are shown next to each other, showing the actual index selections next to the
possible index selections.}\label{fig:plot:eval2:test9}
\end{figure}