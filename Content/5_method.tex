This is the method.

\section{Choice of method} \label{sec:choiceofmethod}
- Explain a query and see what indexes are chosen at what time
- Explain how the database was transferred (quickly)
- The steps:
-- Run executions with parameters
-- Save output to file
-- Parse output to find how data is accessed via an index
-- Identify unique access methods for each relation
- Show pseudocode for each step

\subsection{Choice of databases} \label{sec:choiceofdatabases}
The two database chosen to be evaluated are PostgreSQL and MariaDB. Both of these databases fulfill the following criteria:
\begin{enumerate}
    \item Modern databases that see much use and development;
    \item Open-source projects with code that anyone can read, modify and help develop;
    \item And they implement state-of-the-art algorithms and methods.
\end{enumerate}

In addition to this PostgreSQL is the typical choice for academic evaluation. All research papers mentioned in Chapter \ref{chap:relatedwork} that have implemented new algorithms or modified old ones have done so in PostgreSQL.

MariaDB on the other hand is commonly used in the world of enterprise as it also support MySQL – another database commonly used in enterprise.

Studying these two databases should cover how well a modern state-of-the-art query optimizer performs. In addition, as mentioned in Section \ref{sec:purpose} since both of them are open-source, if one performs better than the other the code can be studied to identify areas of improvement.

\section{Benchmark problems} \label{sec:benchmark}
- TriOptimas, huge and complex

\section{Implementation}
- Programmet implementeras i Python
- Postgres-driver: psycopg

\subsection{PostgreSQL}

\subsection{MariaDB}

\section{Performance analysis}
- Repeat a number of times (f.e.  1000) and see if different indexes are chosen at the same point of execution