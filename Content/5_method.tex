In this chapter, the method used to investigate the problem statement is
presented. First, the choice of method is described, motivating the dataset and
technologies used. Following this the problems used for benchmarking are
presented, including a more in-depth description of the dataset. After these
motivations, the actual implementation details are presented, showing how the
database's are evaluated.

\section{Choice of method}\label{sec:choiceofmethod}
This section will motivate the methods' three primary questions:
\begin{enumerate}
\item What databases are evaluated?
\item What dataset is used to evaluate the databases?
\item How are the databases evaluated?
\end{enumerate}

The following three sections will answer these questions in order, motivating
choice of databases, dataset and implementation.

\subsection{Choice of databases}\label{sec:choiceofdatabases}
The two database chosen to evaluated are PostgreSQL and MariaDB. The choice was
made, as they both are:
\begin{enumerate}
\item Modern databases with widespread use and active development;
\item Open-source projects allowing anyone to read and modify the source code;
\item And they implement state-of-the-art algorithms and methods.
\end{enumerate}

In addition to this they both cover two common use-cases: academia and
enterprises. All research papers mentioned in Chapter~\ref{chap:relatedwork} that have
implemented new algorithms or modified old ones have done so in PostgreSQL.
On the other hand MariaDB is compatible with MySQL, making it a common
alternative for enterprises.

An evaluation of both of these database will give a good indication of the
performance of a modern state-of-the-art query optimizer. Furthermore, as
mentioned in Section~\ref{sec:purpose} since both of them are open-source, if
one performs better than the other the code can be studied to identify areas of
improvement.

\subsection{Choice of dataset}\label{sec:dataset}
The primary focus when selecting the dataset used was to use a dataset which
could capture the complexity of a real-world database and provide a realistic
challenge for the query optimizer. The primary requirements are that the dataset
feature:
\begin{enumerate}
\item Many tables with multiple indexes;
\item Not just indexes on a single row, but also compound indexes covering
  several;
\item Skewed and non-uniform data that is not trivial to correctly estimate
  based on sampling;
\item A large amount of data forcing the database to estimate the data.
\end{enumerate}

The datasets most commonly used for evaluation of database implementations are
TPC-H~\cite{tpc_th}, TPC-DS~\cite{tpc_tha} and more recently
JOB~\cite{leis_2015_how_hgaqor}. None of these datasets however feature the
several indexes on the same table or compound indexes, making the problem of
index selection trivial for all of them.

Instead, the dataset chosen was one taken from the real world: the dataset for
TriOptima's product triReduce. This database fulfill all of the requirements, as
can be seen in Table~\ref{table:dataset}. The metrics of the dataset are
presented in more detail in Section~\ref{sec:benchmark}.

Another important aspect of the dataset is the queries used for evaluation.
Selecting these was done based on the following critera:
\begin{itemize}
\item The tables queried must be sufficiently large to require the cardinality to be
  estimated via sampling;
\item The data most be accessible via one or more index so that the actual index
  selection is not trivial for the query optimizer.
\end{itemize}

The two criteria are not fulfilled for more than a few tables in a database,
reducing the amount of queries relevant for evaluation. However, the queries
that do fulfill the above requirements are also those that are most interesting
to study for a database as they are the ones that will take the most execution time.

\subsection{Choice of implementation}\label{sec:implchoice}
The focus when implementing the tool used to evaluate the databases was to find
a tool that would allow a high-level description of the data transformations
necessary. Additionally the language must be sufficiently stable and be able to
handle potentially large amounts of simultaneous data.

The language chosen that fulfill these requirements is Clojure. Clojure compiles
to bytecode that runs on the JVM, which is stable and well-used. Additionally
the language is well-suited to describing data transformations as it provides
many high-level functions for doing so.

More information regarding Clojure and the tool developed will be presented in
Section~\ref{sec:implementation}, which also shows how some of the data
transformations are done in practice.

\section{Benchmark problems}\label{sec:benchmark}
This section describe the problems used for benchmarking, starting with
specification of the hardware that the tests were ran on. Following this is
first a description of the metrics of the dataset used.

\subsection{Hardware specs}
All evaluation tests were ran on a dedicated computer running only the databases
and tests. The most important part of the hardware is to ensure that there is
sufficient data for both the databases and the results of the tests. For all
evaluations three hard drives were used, one for each database and one for the
tool itself.

The exact specifications are:
\begin{itemize}
\item 2 \textit{Intel® Xeon® Processor E5-2643 (10M Cache, 3.30 GHz, 8.00 GT/s Intel®
    QPI)}, featuring 4 cores each;
\item 1 \textit{Seagate Savvio 15K.3 ST9146853SS 146GB 15000 RPM 64MB Cache SAS 6Gb/s
    2.5"}, used to store the project itself on;
\item 1 \textit{Seagate Constellation ES.3 ST4000NM0023 4TB 7200 RPM 128MB Cache SAS
    6Gb/s 3.5"}, used to store the PostgreSQL database on;
\item And 1 \textit{Seagate Constellation ES ST2000NM0001 2TB 7200 RPM 64MB Cache SAS 6Gb/s
    3.5"}, used to store the MariaDB database on.
\end{itemize}

As a final note it is worth pointing out that the effect of the hardware should
have none, or very little, effect on the query optimizer's plan selection.

\subsection{The dataset}
As detailed in Section~\ref{sec:dataset} the dataset should be sufficiently
complex in terms of indexes, table size and table values.
Table~\ref{table:dataset} presents the number of indexes, the number of rows and
the size in MB of the entire database and all tables involved in the
benchmarking, the names of the tables have anonymized and are referred to by an
identifier such as ``mm'' or ``book''.

\begin{table}
  \begin{center}
    \begin{tabu} {c c c c}
      \toprule
      name & \#index & \#rows & size (MB) \\
      \midrule
      database total & 1130 & 305 & 1165290 \\
      mm & 6 & 64882651 & 9448 \\
      book & 6 & 51709 & 10 \\
      resamb & 3 & 40598 & 5 \\
      cmm & 2 & 17335822 & 1219 \\
      cmt & 9 & 52808814 & 12811 \\
      t & 35 & 115851469 & 92633 \\
      est & 32 & 33726190 & 19434 \\
      ct & 23 & 115751571 & 72320 \\
      mt & 9 & 21721256 & 4284 \\
      \bottomrule
    \end{tabu}
    \caption[The metrics for the dataset]{The metrics for the dataset used for
      evaluation of the databases. Both the metrics for the entire database and
      those of individual tables are shown. Note that the table names have been
      anonymized and are only referred to by an identifier.}\label{table:dataset}
  \end{center}
\end{table}

The original dataset was stored in a MySQL database, which was ported to
PostgreSQL and MariaDB. MariaDB is made as an add-on to MySQL and thus required
no specific porting, the data was just copied into a fresh install of MariaDB
using a \textit{mysqldump}. For PostgreSQL
\textit{py-mysql2pgsql}~\cite{philipsoutham_p} was used to create a dump of the
MySQL database with all MySQL specific data types converted to their most
similar PostgreSQL equivalents. The data was then read into a fresh install of PostgreSQL.

In the copying process all indexes and relations were maintained, thus
maintaining the same metrics for both of the copies as for the original.

\subsection{The queries}\label{sec:queries}
To evaluate the databases only one query was used. The
reason is that, as described in Section~\ref{sec:dataset}, there are few tables that can be
involved in the query as they must fulfill the important critera in terms of
indexes and size. As such one query covering all of the most complex tables were
written, this query can be considered to be the most complex query for the
dataset. Subsets are then used to simulate simpler scenarios.

The original query used for evaluation can be seen in
Figure~\ref{fig:sql:query1}, the table names are once again anonymized. For the
metrics of the tables involved see Figure~\ref{table:dataset}. The variations of
the query simply remove one or more tables involved.

\begin{figure}[ht]
\begin{minted}[breaklines]{clojure}
SELECT *
FROM ct JOIN t JOIN mt JOIN mm JOIN book JOIN cmt JOIN cmm JOIN est JOIN resamb
WHERE ct.key = :KEY
\end{minted}
    \caption[ The original query used for evaluation ]{The query used for
      evaluation, simplified and anonymized. The tables are joined on rows in
      common.}\label{fig:sql:query1}
\end{figure}

\section{Implementation}\label{sec:implementation}
This section will cover the implementation details of the tool used for
evaluation. The section starts with a general overview of the process of
evaluation, breaking it down into steps. Following this is a description of the
tool developed, showing how it is used and what technologies are used.
The next three sections then cover the steps of evaluation and specific
implementation details for each. Finally implementation details are provided for
PostgreSQL and MariaDB.

Evaluating a database for a given query can be broken down into three primary steps:
\begin{enumerate}
\item Repeatedly executing the query with different sample sizes to generate
  query plans;
\item Parsing the query plans to find what access methods are used for all
  relations;
\item And finally analyzing the parsed plans to find the number of unique access
  methods used for each relation;
\end{enumerate}

These three steps must be executed in order, but they can be executed
independently of each other; it is possible to only generate plans at a time and
save them for later parsing and analysis.

\subsection{Overview of the tool}
The tool was implemented so as to be executed from the command line, providing
all the necessary parameters via flags. As mentioned in
Section~\ref{sec:implchoice} the tool was developed using Clojure and is
therefore typically ran via Leiningen, as shown in Figure~\ref{fig:cmd:runtool1}.

\begin{figure}[ht]
  \begin{minted}[breaklines]{bash}
    lein run steps='generate parse analyze' query=queryid repetitions=100 samplesizes='10 100' --database=postgresql
  \end{minted}
  \caption[Using the tool to generate, parse and analyze a query]{An example of
    using the tool to generate, parse and analyze a query with some given
    parameters, such as the sample sizes to use.}\label{fig:cmd:runtool1}
\end{figure}

As the steps can be executed one at a time, the results for each are saved and
the tool allows the execution of only a specific set of steps at a time. An
example of this can be seen in Figure~\ref{fig:cmd:runtool2} where a previously
generated plan is parsed and analyzed.

\begin{figure}[ht]
  \begin{minted}[breaklines]{bash}
    lein run plans/xxx-000000000 steps='parse analyze'
  \end{minted}
  \caption[Using the tool to parse and analyze a previously generated plan]{An
    example of how the tool can be used to parse and analyze a previously
    generated file containing query plans.}\label{fig:cmd:runtool2}
\end{figure}

The full code of the tool can be seen in Appendix~\ref{appendix:sourcecode} and is
open-source and can be found at~\cite{barksten_mbark_m}. The repository has
further documentation regarding project structure etc.

\subsection{Generating plans}\label{sec:generatingplans}
The task of generating query plans can be broken down in the following steps:
\begin{enumerate}
\item Set the statistics target for the database;
\item Generate new statistics for all tables involved in the query used for evaluation;
\item Find the query plan for each parameter value of the query.
\end{enumerate}
These steps can then be repeated a number of times to ensure that all query plans are
found.

The most relevant parts of the code used to generate plans can be found in
Figure~\ref{fig:clj:generating}. In the Figure two functions can be seen:
\clj{generate-plans} and \clj{sample-and-query}, additional functions are
referenced but not seen, for a full reference see Appendix~\ref{appendix:sourcecode}.

The generation step is handled by \clj{generate-plans}, which is provided the
options for the evaluation. This function will then for each sample size,
repeated the number of times specified call \clj{sample-and-query}.

The \clj{sample-and-query} function will in turn perform all of the steps
outlined above; set the statistics target and generate new statistics via
\clj{resample-with!} and then find the query plan for all possible parameter
values via repeated calls to \clj{explain-query}.

Each plan found is directly saved to file for later use in parsing.

\begin{figure}[ht]
  \begin{minted}[breaklines]{clj}
(defn- sample-and-query [save-plan options]
  (resample-with! options)
  (doseq [param param-range]
    (save-plan (explain-query options param))))

(defn generate-plans [opts save-plan]
  (j/with-db-connection [db-con (opts->db-info opts)]
    (doseq [sample-size (:samplesizes opts)]
      (dotimes [i (:repetitions opts)]
       (sample-and-query save-plan
                         (assoc
                          opts
                          :sample-size sample-size
                          :connection db-con))))))
   \end{minted}
   \caption[The clojure code to generate a query]{The relevant parts of the
     Clojure code used to generate the query plans. Some function definitions
     have been removed to improve readability, see the
     Appendix~\ref{appendix:sourcecode} for the full code}
\label{fig:clj:generating}
\end{figure}

\subsection{Parsing the plans}\label{sec:parsing}
Parsing the generated plans is done by simply stripping all information but the
access methods from the query plans, after this is done the access methods are
grouped by what relation they access. The code for the parsing can be seen in
Figure~\ref{fig:clj:parsing}.

Parsing a plan is done with the function \clj{parse-plan}, which will find all
access methods with \clj{find-relation-accesses} and group these by their relation
with \clj{group-by-relation}. Finding the access methods in the query plan is
done by traversing the plan as a tree and calling \clj{save-if-relation-access}
on each value --- storing it if it described an access.

\begin{figure}[ht]
\begin{minted}[breaklines]{clj}
(defn- save-if-relation-access [db-id o]
  (if (and (map? o) (contains? o db-id))
    (swap! relation-accesses conj o))
  o)

(defn- find-relation-accesses [db-id plan]
  (reset! relation-accesses [])
  (postwalk #(save-if-relation-access db-id %) plan)
  @relation-accesses)

(defn- group-by-relation [db-id accesses]
  (group-by
   #(get % db-id)
   accesses))

(defn parse-plan [db plan]
  (let [db-id (access-key db)]
    (group-by-relation db-id (find-relation-accesses db-id plan))))
   \end{minted}
   \caption[The clojure code to parse a query]{The clojure code used to parser
     the query plan output from the generation step.}
\label{fig:clj:parsing}
\end{figure}

Each generated plan is parsed and the new plan saved to another file. The main
purpose is to transform the data into something more easily analyzed.
Additionally the size of the query plans are reduced in size, reducing the
time taken to analyze all plans.

\subsection{Analyzing the plans}\label{sec:analyzingplans}
Analyzing the plans is done by merging all access methods found when parsing the
plans, keeping the distinct methods for each relation. The code for analysis can
be seen in Figure~\ref{fig:clj:analyzing}.

Analyzing all generated plans for a sample size is done by the function
\clj{analyze-plans}, which is provided an identifier for the database evaluated,
a function to read the next plan and the total number of plans to read. The
analysis is done by generating map, reading the next plan and merging it with
the previous one. If the same relation is found in both (which will always be
the case) the function \clj{conj-distinct} will add only the new access methods
found that are distinct from those previous found. Only the index used is
considered in terms of two plans being distinct from each other.

\begin{figure}[ht]
\begin{minted}[breaklines]{clj}
(defn conj-distinct [f x y]
  (reduce
   (fn [coll v]
     (if (some #(= (f %) (f v)) coll)
       coll
       (conj coll v)))
   x y))

(defn analyze-plans [db next-plan plans-to-read]
  (loop [m {} plans-left plans-to-read]
    (if (zero? plans-left)
      m
      (recur
       (merge-with
        #(conj-distinct (fn [access] (get access (idx-key db)))
                        %1 %2)
        m (next-plan))
       (dec plans-left)))))
   \end{minted}
   \caption[The clojure code to analyze a query]{The Clojure code used to
     analyze the parsed output. The code will merge the maps generated, only
     keeping the unique access methods for each relation.}
\label{fig:clj:analyzing}
\end{figure}

The analysis is done for each sample size, and the resulting map of relation to
access methods is saved for study.

\subsection{PostgreSQL}\label{sec:postgresql}
In PostgreSQL the SQL commands used are quite straightforward, one is used to
delete all previously gathered statistics, one to set the statistics target and
finally an \sql{ANALYZE} is called for each table involved in the query being
evaluated. The specific commands used can be seen in
Figure~\ref{fig:sql:pganalyze}, the value of the sample size is provided as a
host variable, as it will depend on the options used when evaluating a database
and query.

\begin{figure}[ht]
\begin{minted}[breaklines]{postgresql}
  DELETE FROM pg_statistics;
  SET default_statistics_target TO :SAMPLE_SIZE;
  ANALYZE table1;
  ANALYZE table2;
\end{minted}
\caption[The SQL commands used to resample inPostgreSQL.]{The SQL commands used
  to first delete all statistics in PostgreSQL, set the statistics target and
  finally analyze all tables involved in the query.}
\label{fig:sql:pganalyze}
\end{figure}

\subsection{MariaDB}\label{sec:mariadb}
In MariaDB the commands used to resample the data are storage engine specific,
in this case InnoDB supports the storage. It is worth noting that InnoDB only
supports MariaDB with statistics, it is then MariaDB that performs the actual
query optimization.

It is necessary in InnoDB to ensure that the statistics used are those
generated, to do this no persistent statistics are saved. Furthermore, no data
is deleted between resampling as it is neither possible nor necessary, the old
statistics are overwritten by the new. Finally the tables are all analyzed in
one single \sql{ANALYZE} call.

The specific commands used can be seen in Figure~\ref{fig:sql:resamplemdb}.

\begin{figure}[ht]
\begin{minted}[breaklines]{mysql}
  SET GLOBAL innodb_stats_persistent='OFF';
  SET GLOBAL innodb_stats_auto_recalc='OFF';
  SET GLOBAL innodb_stats_transient_sample_pages = :SAMPLE_SIZE;
  ANALYZE TABLE table1, table2;
\end{minted}
\caption[The SQL commands used to resample in MariaDB.]{The SQL commands used to
first ensure that MariaDB will not use some other stats than those we gather,
then set the statistics target and finally analyze all tables involved in the query.}
\label{fig:sql:resamplemdb}
\end{figure}